% LaTeX set-up adapted from a template by Alan T. Sherman (9/9/98)
%%%%%%%%%%%%%%%%%%%%%%%%%%%%%%%%%%%%%%%%%%%%%%%%%%%%%%%%%%%%%%%%%%%%%%
\documentclass[12pt]{article}
\usepackage{amsmath}
\usepackage{amsfonts}

% Set the margins
%
\setlength{\textheight}{8.5in}
\setlength{\headheight}{.25in}
\setlength{\headsep}{.25in}
\setlength{\topmargin}{0in}
\setlength{\textwidth}{6.5in}
\setlength{\oddsidemargin}{0in}
\setlength{\evensidemargin}{0in}

%%%%%%%%%%%%%%%%%%%%%%%%%%%%%%%%%%%%%%%%%%%%%%%%%%%%%%%%%%%%%%%%%%%%%%%
% Macros

% Math Macros.  It would be better to use the AMS LaTeX package,
% including the Bbb fonts, but I'm showing how to get by with the most
% primitive version of LaTeX.  I follow the naming convention to begin
% user-defined macro and variable names with the prefix "my" to make it
% easier to distiguish user-defined macros from LaTeX commands.
%
\newcommand{\myfunction}[3] {${#1} : {#2} \rightarrow {#3}$ }
\newcommand{\myzrfunction}[1] {\myfunction{#1}{{\myZ}}{{\myR}}}
\renewcommand{\iff}{\Leftrightarrow}

% Formating Macros %

\newcommand{\myheader}[4] {\vspace*{-0.5in} \noindent{#1} \hfill {#2} \newline \noindent{#3} \hfill {#4} \noindent \rule[8pt]{\textwidth}{1pt} \vspace{1ex} }  % end \myheader 

\newcommand{\myalgsheader}[0] {\myheader{MASc Research Weekly Update}{UTIAS SRG}{Anton Rubisov}{September 14, 2014}} 

% Running head (goes at top of each page, beginning with page 2.
% Must precede by \pagestyle{myheadings}.
\newcommand{\myrunninghead}[2] {\markright{{\it {#1}, {#2}}}}

\newcommand{\myrunningalgshead}[2] {\myrunninghead{MAT315S Number Theory}{{#1}}}

\newcommand{\myrunninghwhead}[1] {\myrunningalgshead{Solutions to Tutorial Problems {#1}}}

\newcommand{\mytitle}[1] {\begin{center} {\large {\bf {#1}}} \end{center}}

\newcommand{\myhwtitle}[1] {\begin{center} {\large {\bf Weekly Update {#1}}} \end{center}}

\newcommand{\mysection}[1] {\noindent {\bf {#1}}}

\newcommand\mat[1]{\boldsymbol{#1}}

%%%%%% Begin document with header and title %%%%%%%%%%%%%%%%%%%%%%%%%

\begin{document}
\myalgsheader
\pagestyle{plain}
%\myhwtitle{1}
\bigskip

%%%%% Begin body %%%%%%%%%%%%%%%%%%%%%%%%%%%%%%%%%%%%%%%%%%%%%%%%%%%

\section*{Framework}
Our goal is to use the dynamics of the Limit Order Book (LOB) as an indicator for
high-frequency stock price movement, thus enabling statistical arbitrage. Formally, 
we will the study limit order book imbalance process, $I(t)$, and the stock price process,
$S(t)$, and attempt to establish a stochastic relationship $\dot{S} = f(S,I,t)$.

LOB imbalance, the ratio of limit order volumes between the bid and ask sides, is calculated
as $I(t) = \dfrac{V_b(t) - V_a(t)}{V_b(t) + V_a(t)} \in [-1,1]$. 

\section*{Modeling $I(t)$: Continuous Time Markov Chain}
Instead of modeling imbalance directly, an alternative approach is to discretize imbalance
into subintervals (called bins), and model a stochastic process that tracks which
bin $I(t)$ falls into. A naive model that can be employed is a continuous-time Markov chain (CTMC).

Let $Z(t)$ be a CTMC taking values in $\left\lbrace 1,\dots , K \right\rbrace$, and having
infinitessimal generator matrix $\mat{G}$.\footnote{Define the terms 
$P_{ij}(t) = P \lbrace Z(t) = j | Z(0) = i \rbrace$. Then the matrices 
$\mat{P}(t) = \lbrace P_{ij}(t) \rbrace$ and $\mat{G}$ satisfy
$\dot{\mat{P}}(t) = \mat{G} \cdot \mat{P}(t)$, and hence $\mat{P}(t) = e^{\mat{G}t}$} 
Conditional on being in some regime $k$, the arrival of buy and sell market orders follow
independent Poisson processes with intensities $\lambda_k^\pm$ (and are hence Markov-modulated
Poisson processes), where $\lambda_k^+$ ($\lambda_k^-$) is the rate of arrivals of market sells (buys).

Given a set of observations of buy/sell market orders and regime switches, we previously derived a
maximum likelihood estimation (MLE) for both the entries of $\mat{G}$ and the values $\lambda_k^\pm$. 
Where $\mat{G} = \lbrace q_{ij} \rbrace \in \mathbb{R}^{K \times K}$, the $q_{ij}$ represent the
transition rates from bin $i$ to $j$ for $i \neq j$, and $q_{ii} = - \sum_{j \neq i} q_ij$ such
that the rows sum to $0$. We found that:
$$ \hat{q}_{ij} = \dfrac{N_{ij}(T)}{H_i(T)} $$
where
\begin{align*} 
N_ij(T) & \equiv \text{number of transitions from bin $i$ to $j$ up to time $T$} \\
H_i(T) & \equiv \text{holding time in bin $i$ up to time $T$}
\end{align*}

Similarly, for the Poisson process intensities $\lambda_k^\pm$, we found:
$$ \hat{\lambda}_k^\pm = \dfrac{N_k^\pm(T)}{H_k(T)} $$
where
\begin{align*} 
N_ij(T) & \equiv \text{number of market orders in bin $k$ up to time $T$} \\
H_i(T) & \equiv \text{holding time in bin $k$ up to time $T$}
\end{align*}

\subsection*{Calibrating a CTMC}
We estimated parameters for a CTMC on a day's worth of LOB data.
Using these parameters, we generated sample paths of the imbalance bins as well as arrival of market orders,
and re-estimated parameters along the sample paths. By doing this for 10,000 paths we obtained histograms for
the parameters (the individual entires of $\mat{G}$ as well as the intensities $\lambda_k^\pm$). 

Using data for \texttt{ORCL} from 2013-05-15, averaging imbalances over a 100ms window, and taking the number
of bins $K=3$, we obtained the following mean values for the parameters:

$$ \mat{G} = \begin{pmatrix} -0.112 & 0.098 & 0.0122 \\
							0.099 & -0.21 & 0.111 \\
							0.0115 & 0.112 & -0.1235 \end{pmatrix}$$ \\
$$ \mat{\lambda} = \bordermatrix{  & k=1    & k=2   & k=3   \cr
								+ & 0.121  & 0.081 & 0.048 \cr
								- & 0.0263 & 0.062 & 0.153 } $$			

\section*{Modeling $I(t)$: Hidden Markov Model}
Knowing that trading intensity varies over the course of a day, and suspecting possible other sources
of double-stochasticity, we tried to fit a Hidden Markov Model (HMM) to the imbalance process $I(t)$.
An HMM is specified with the following parameters:
\begin{itemize}
\item Underlying hidden states $S = \lbrace S_1, \dots, S_N \rbrace$
\item Observation symbols $V = \lbrace v_1, \dots, v_M \rbrace$ (a discrete alphabet)
\item State transition probabilities $\mat{A} = \lbrace a_{ij} \rbrace$, where $a_{ij} = P \lbrace q_{t+1} = S_j \;| \; q_t = S_i \rbrace$
\item Observation symbol probability distribution $\mat{B} = \lbrace b_j (k) \rbrace$, \\
where $b_{j} (k) = P \lbrace v_k \mbox{ at time } t \; | \; q_t = S_j \rbrace$
\item Initial state distribution $\mat{\pi} = \lbrace \pi_i \rbrace$
\end{itemize}

A short-hand for the parameter space is $\lambda = (\mat{A},\mat{B},\mat{\pi})$. \\

Fitting an HMM to an observation sequence $O = O_1 O_2 \dots O_T$ can be achieved via the Baum-Welch
algorithm (or equivalently the Expectation-Modification algorithm), which seeks to adjust $\lambda$ 
to maximize $P(O \; | \; \lambda)$. There is no analytical solution for an optimal $\lambda$, and in
fact no way of estimating $\lambda$ given a finite $O$. However, the Baum-Welch algorithm allows us
to \underline{locally} optimize $P(O \; | \; \lambda)$.

\subsection*{Calibrating an HMM}
We ran the Baum-Welch algorithm (using the pre-packaged \texttt{hmmtrain} function in MATLAB) on the
same data that was used for the CTMC calibration. Using either 2 or 3 hidden states, and between 3 
and 5 imbalance bins, the HMM training algorithm would converge in {\raise.17ex\hbox{$\scriptstyle\sim$}}13
steps, and yield a matrix $\mat{A}$ (hidden state transitions) with values 
{\raise.17ex\hbox{$\scriptstyle\sim$}}1 on the diagonals and {\raise.17ex\hbox{$\scriptstyle\sim$}}0 
on the off-diagonals. 

We discovered that increasing the imbalance averaging time (which affects the observation data fed to the
training algorithm) has reducing the diagonal values away from 1. As the averaging time approached 60s, the
diagonal values converged to about 0.85.

\section*{Next Steps}
\begin{enumerate}
\item Run cross-validation on the old CTMC imbalance model, also varying the averaging time.
\item Check for a unit root in the imbalance time series using the augmented Dickey-Fuller test, after transforming the data using the logit function.
\item Instead of running a HMM where the hidden state informs the observable imbalance, try having the hidden state affect the transition matrix between imbalance states. 
\item Consider a CTMC where the state is actually the pair $(I_{k-1}, I_k)$, with a $k^2 \times k^2$ transition matrix. Cross-validate and compare with regular CTMC.
\item Same as above but with HMM. 
\end{enumerate}

%%% End solution and document %%%%%%%%%%%%%%%%%%%%%%%%%%%%%%%%%%%%

\end{document}
