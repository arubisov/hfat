% LaTeX set-up adapted from a template by Alan T. Sherman (9/9/98)
%%%%%%%%%%%%%%%%%%%%%%%%%%%%%%%%%%%%%%%%%%%%%%%%%%%%%%%%%%%%%%%%%%%%%%
\documentclass[12pt]{article}
\usepackage[a4paper,bindingoffset=0.2in,left=0.75in,right=0.75in,top=1in,bottom=1in,footskip=.25in]{geometry}
\usepackage{amsmath}
\usepackage{amsfonts}
\usepackage{tabularx}
\usepackage{float}
\usepackage{color}
\usepackage{multicol}

\usepackage{algorithm}
\usepackage{algpseudocode}

% Set multicol stuff
\setlength{\columnseprule}{1pt}
\def\columnseprulecolor{\color{blue}}

% Set table floating
% use \begin{table}[H] to fix its position.
\restylefloat{table}

%%%%%%%%%%%%%%%%%%%%%%%%%%%%%%%%%%%%%%%%%%%%%%%%%%%%%%%%%%%%%%%%%%%%%%%
% Macros

% Math Macros.  It would be better to use the AMS LaTeX package,
% including the Bbb fonts, but I'm showing how to get by with the most
% primitive version of LaTeX.  I follow the naming convention to begin
% user-defined macro and variable names with the prefix "my" to make it
% easier to distiguish user-defined macros from LaTeX commands.
%
\newcommand{\myfunction}[3] {${#1} : {#2} \rightarrow {#3}$ }
\newcommand{\myzrfunction}[1] {\myfunction{#1}{{\myZ}}{{\myR}}}
\renewcommand{\iff}{\Leftrightarrow}
\newcommand{\pluseq}{\mathrel{+}=}
\newcommand{\mineq}{\mathrel{-}=}


% Formating Macros %

% header
\newcommand{\myheader}[4] {\vspace*{-0.5in} \noindent{#1} \hfill {#2} \newline \noindent{#3} \hfill {#4} \noindent \rule[8pt]{\textwidth}{1pt} \vspace{1ex} }
\newcommand{\myalgsheader}[0] {\myheader{MASc Research Weekly Update}{UTIAS SRG}{Anton Rubisov}{\today}} 
% end header

\newcommand\undermat[2]{% http://tex.stackexchange.com/a/102468/5764
  \makebox[0pt][l]{$\smash{\underbrace{\phantom{%
    \begin{smallmatrix}#2\end{smallmatrix}}}_{\text{$#1$}}}$}#2}
    
\newcommand\overmat[2]{%
  \makebox[0pt][l]{$\smash{\overbrace{\phantom{%
    \begin{smallmatrix}#2\end{smallmatrix}}}^{\text{$#1$}}}$}#2}
    
    
\newcommand\mat[1]{\boldsymbol{#1}}


%%%%%% Begin document with header and title %%%%%%%%%%%%%%%%%%%%%%%%%

\begin{document}
\myalgsheader
\pagestyle{plain}
{\begin{center} {\large {\bf Limit Order Book Dynamics}} \end{center}}
\bigskip

%%%%% Begin body %%%%%%%%%%%%%%%%%%%%%%%%%%%%%%%%%%%%%%%%%%%%%%%%%%%

\begin{quote}
Our goal is to use the dynamics of the Limit Order Book (LOB) as an indicator for
high-frequency stock price movement, thus enabling statistical arbitrage. Formally, we will the study limit order book imbalance process, $I(t)$, and the stock price process, $S(t)$, and attempt to establish a stochastic relationship $\dot{S} = f(S,I,t)$. We will then attempt to derive an optimal trading strategy based on the observed relationship.
\end{quote}

\section*{Recap Next Steps}
\begin{enumerate}
\item Complete in-sample backtesting of the `naive' trading strategies.
\item Formulate stochastic control problem 
\item Extra Reading: Bellman Equations, MDP, Partially Observable MDP
\end{enumerate}

\section*{In-Sample Backtesting of Naive Trading Strategies}
As a refresher: 

We are a considering a CTMC for the joint distribution $(I(t), \Delta S(t))$ where $I(t) \in \lbrace 1,2,\dots,\#_{bins} \rbrace$ is the bin corresponding to imbalance averaged over the interval $[t-\Delta t_I, t]$, and $\Delta S(t) = \text{sign}(S(t+\Delta t_S)-S(t)) \in \lbrace -1, 0, 1 \rbrace$, considered individually for the best bid and best ask prices. The pair $(I(t), \Delta S(t))$ was then reduced into one dimension with a simple encoding.

From the resulting timeseries we estimated a generator matrix $\mat{G}$ and used it to obtain a one-step transition probability matrix $\mat{P} = e^{\mat{G}\Delta t_I}$. The entries of $\mat{P}$ contain the conditional probabilities $\mathbb{P}\left[ \rho_{curr}, \Delta S_{curr} \; | \; \rho_{prev}, \Delta S_{prev} \right]$, from which we can solve for the probability of now seeing a given price change ($\Delta S_{curr}$) conditional on the current imbalance, the previous imbalance, and the previous price change.

For example, one such conditional probability matrix $\mat{P_C}$ (using 3 imbalance bins) was:

$$  \begin{smallmatrix}
    \Delta S_n < 0 \rightarrow \\
    \Delta S_n = 0 \rightarrow \\
    \Delta S_n > 0 \rightarrow
  \end{smallmatrix}
  \left [
    \begin{smallmatrix}
\overmat{\rho_n = 1}{.67 & .05 & .04 & .01 & .03 & .04 & .00 & .05 & .05} & \overmat{\rho_n = 2}{.02 & .50 & .12 & .01 & .00 & .02 & .05 & .01 & .02} & 
\overmat{\rho_n = 3}{.00 & .00 & .52 & .00 & .01 & .00 & .00 & .00 & .00} \\
.33 & .95 & .96 & .99 & .97 & .96 & .41 & .93 & .95 & .96 & .49 & .87 & .98 & .99 & .97 & .91 & .48 & .96 & .98 & .95 & .47 & .95 & .96 & .93 & .98 & .88 & .34 \\ \undermat{\Delta S_{n-1} < 0}{.00 & .00 & .00} & 
\undermat{}{.00 & .00 & .00} & 
\undermat{\Delta S_{n-1} > 0}{.58 & .02 & .00} & 
\undermat{}{.02 & .01 & .00} & 
\undermat{\Delta S_{n-1} = 0}{.01 & .01 & .01} & 
\undermat{}{.05 & .51 & .01} & 
\undermat{}{.02 & .04 & .01} & 
\undermat{}{.05 & .03 & .02} & 
\undermat{}{.02 & .12 & .66}
    \end{smallmatrix}
  \right ]
$$

\vspace{0.5cm}
Immediately evident from $\mat{P_C}$ is that in most cases we are expecting no price change. In fact, the only cases in which the probability of a price change is $>0.5$ show evidence of \textit{momentum}; for example, the way to interpret the value in row 1, column 1 is: if $\rho_{prev} = \rho_{curr} = 1$ and previously we saw a downward price change, then we expect to again see a downward price change. In fact, the best way to summarize the matrix is:

$$ \mathbb{P} \left[ \Delta S_{curr} = \Delta S_{prev} \; | \; \rho_{prev} = \rho_{curr} \right] > 0.5 $$

We backtested a number of naive trading strategies, outlined below, based on this observation.

\begin{algorithm}
\caption{Naive Trading Strategy}
\begin{algorithmic}[1]
\State $cash = 0$
\State $asset = 0$
\For{$t=2 \; : \; \texttt{length}(timeseries)$}
	\If {$\mathbb{P} \left[ \Delta S_{curr} < 0 \; | \; \rho_{curr}, \rho_{prev}, \Delta S_{prev} \right] > 0.5$}
	\State $cash \pluseq data.BuyPrice(\textit{t})$
	\State $asset \mineq 1$
	\ElsIf {$\mathbb{P} \left[ \Delta S_{curr} > 0 \; | \; \rho_{curr}, \rho_{prev}, \Delta S_{prev} \right] > 0.5$}
	\State $cash \mineq data.SellPrice(\textit{t})$	
	\State $asset \pluseq 1$
	\EndIf
\EndFor
\If {$asset > 0$} 
\State $cash \pluseq asset \times data.BuyPrice(\textit{t})$
\ElsIf {$asset < 0$} 
\State $cash \pluseq asset \times data.SellPrice(\textit{t})$	
\EndIf
\end{algorithmic}
\end{algorithm}

\begin{algorithm}
\caption{Naive+ Trading Strategy}
\begin{algorithmic}[1]
\State $cash = 0$
\State $asset = 0$
\For{$t=2 \; : \; \texttt{length}(timeseries)$}
	\If {$\mathbb{P} \left[ \Delta S_{curr} < 0 \; | \; \rho_{curr}, \rho_{prev}, \Delta S_{prev} \right] > 0.5$}
	\State $cash \pluseq data.BuyPrice(\textit{t})$
	\State $asset \mineq 1$
	\ElsIf {$\mathbb{P} \left[ \Delta S_{curr} > 0 \; | \; \rho_{curr}, \rho_{prev}, \Delta S_{prev} \right] > 0.5$}
	\State $cash \mineq data.SellPrice(\textit{t})$	
	\State $asset \pluseq 1$
	\EndIf
\EndFor
\If {$asset > 0$} 
\State $cash \pluseq asset \times data.BuyPrice(\textit{t})$
\ElsIf {$asset < 0$} 
\State $cash \pluseq asset \times data.SellPrice(\textit{t})$	
\EndIf
\end{algorithmic}
\end{algorithm}


%%% End solution and document %%%%%%%%%%%%%%%%%%%%%%%%%%%%%%%%%%%%

\end{document}
