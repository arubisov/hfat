% LaTeX set-up adapted from a template by Alan T. Sherman (9/9/98)
%%%%%%%%%%%%%%%%%%%%%%%%%%%%%%%%%%%%%%%%%%%%%%%%%%%%%%%%%%%%%%%%%%%%%%
\documentclass[12pt]{article}
\input{/home/anton/Documents/latex/LaTeXHeader.tex} % put hard path to header.
\usepackage{soul}
\newtheorem{theorem}{Theorem}
\usepackage{hyperref}
\allowdisplaybreaks

%%%%%% Begin document with header and title %%%%%%%%%%%%%%%%%%%%%%%%%

\begin{document}
\mymascheader
\pagestyle{plain}
{\begin{center} {\large {\bf High-Frequency Algorithmic Trading \\ with Momentum and Order Imbalance}} \end{center}}
\bigskip

%%%%% Begin body %%%%%%%%%%%%%%%%%%%%%%%%%%%%%%%%%%%%%%%%%%%%%%%%%%%

\begin{quote}
My goal is to establish and solve the stochastic optimal control problem that 
captures the momentum and order imbalance dynamics of the Limit Order Book 
(LOB). The solution will yield an optimal trading strategy that will permit
statistical arbitrage of the underlying stock, which will then be backtested on
historical data.
\end{quote}

\section*{Progress Timeline}
\begin{table}[H]
\renewcommand\arraystretch{1.4}\arrayrulecolor{LightSteelBlue3}
\newcommand{\foo}{\color{LightSteelBlue3}\makebox[0pt]{\textbullet}\hskip-0.5pt\vrule width 1pt\hspace{\labelsep}}
\newcommand{\fooo}{\color{LightSteelBlue3}\hskip-0.5pt\vrule width 1pt\hspace{\labelsep}}
\begin{tabular}{@{\,}r <{\hskip 2pt} !{\foo} >{\raggedright\arraybackslash}p{9cm} !{\fooo} >{\raggedright\arraybackslash}p{5cm}} 
\multicolumn{1}{@{\,}r <{\hskip 9pt}}{DATE} & \multicolumn{1}{l}{THESIS} & \multicolumn{1}{l}{STA4505} \\
\hline
\st{Dec 2014} & \st{Complete CTMC calibration} \\
\st{Dec 2014} & \st{Backtest naive strategies based on CTMC} & \\
\st{Jan-May} & \st{Study stochastic controls: ECE1639, STA4505} & \\
\st{Jun 5} & \st{Establish models} & \st{Exam Study} \\
\st{Jun 12} & \st{Establish performance criteria} & \st{Exam Study} \\
\st{Jun 15} & \st{Derive DPP/DPE} & \st{EXAM} \\
\st{Jun 19} & \st{Derive DPP/DPE} &  \\
\st{Jun 26} & \st{Derive continuous-time equations} & \\
\st{Jul 3} & \st{Derive discrete-time equations} & \\
\st{Jul 10} & \st{Set up MATLAB numerical integration} &  \\
Jul 17 & Integrate functions and plot dynamics & \st{Integrate and analyze} \\
Jul 24 & More dynamics, and calib/choose parameters & \\
Jul 31 & Backtest on historical data & \st{Simulate results} \\
Aug 7 & More backtesting, comparing with previous & \\
Aug 14 & Dissertation writeup / buffer & Project writeup \\
Aug 21 & Dissertation writeup / buffer &  \\
Aug 28 & Dissertation writeup & Presentation \\
\end{tabular}
\end{table}

\newpage
\section*{Whiteboard Inspirational Quote of the Week}
\begin{quote}
\textit{I'm sorry to say that the subject I most disliked was mathematics. I have thought about it. I think the reason was that mathematics leaves no room for argument. If you made a mistake, that was all there was to it.} \\-― Malcolm X 
\end{quote} 



\section*{The Academic Week in Review}
Reminder of our processes (a little bit of abuse of notation going on):\\
$\bz_k = (\rho_k, \Delta_k)$ - 2-D time-homogenous Markov Chain with transition probabilities $P_{ij}$, where $\rho_k \in \Gamma$ and $\Gamma$ represents the set of imbalance bins, and $\Delta_k = \sgn(s_k - s_{k-1}) \in \{-1,0,1\}$. \\
State $\vec{x}_{k} = \begin{pmatrix}
x_k \\
s_k \\
\bz_k \\
q_k 
\end{pmatrix} \qquad \begin{matrix}
\text{cash} \\
\text{stock price} \\
\text{Markov chain state, as above} \\
\text{inventory} \\
\end{matrix}$ \\
Control $\vec{u}_{k} = \begin{pmatrix}
\delta_k^+ \\
\delta_k^- \\
M_k^+ \\
M_k^-
\end{pmatrix} \qquad \begin{matrix}
\text{bid posting depth} \\
\text{ask posting depth} \\
\text{buy MO - binary control} \\
\text{sell MO - binary control} \\
\end{matrix}$ \\
Random $\vec{w}_{k} = \begin{pmatrix}
K_k^+ \\
K_k^- \\
\omega_k
\end{pmatrix} \qquad \begin{matrix}
\text{other agent buy MOs - binary} \\
\text{other agent sell MOs - binary} \\
\text{random variable uniformly distributed on [0,1]} \\
\end{matrix}$ \\

We'll write the evolution of the Markov chain as a function of the current state and a uniformly distributed random variable $\omega$:\footnote{Borrowed from ECE1639 notes.}
\begin{equation}
\bz_{k+1} = T(\bz_k, \omega_k) = \sum_{i=0}^{|\Gamma|} i \cdot \indicator_{\left( \sum_{j=0}^{i-1} P_{\bz_k,j} , \sum_{j=0}^{i} P_{\bz_k,j} \right]} (\omega_k)
\end{equation}
Here $\indicator_A(\omega) = \begin{cases} 1 & \text{if } \omega \in A \\
0 & \text{if } \omega \not\in A
\end{cases}$, and hence $Z_{k+1}$ is assigned to the value $i$ for which $\omega_k$ is in the indicated interval of probabilities.

Our Markovian state evolution function $f$, given by $\vec{x}_{k+1} = f \left( \vec{x}_{k},\vec{u}_{k}, \vec{w}_{k} \right)$, can be written explicitly as
\begin{equation}
\label{eq:discretestateevolution}
\begin{pmatrix}
x_{k+1} \\
s_{k+1} \\
\bz_{k+1} \\
q_{k+1} 
\end{pmatrix} = \begin{pmatrix}
x_k + (s_k + \pi + \delta_k^-)L_k^- - (s_k - \pi - \delta_k^+)L_k^+ + (s_k - \pi)M_k^- - (s_k + \pi)M_k^+ \\
s_{k} + \eta_{k,\bz_k} T(\bz_k, \omega_k)^{(2)} \\
T(\bz_k, \omega_k) \\
q_{k} + L_k^+ - L_k^- + M_k^+ - M_k^-
\end{pmatrix}
\end{equation}
Clearly, the cash process at a subsequent is equal to the cash at the previous step, plus the costs of profits of executing market or limit orders. There are two noteworthy observations regarding this formulation of the evolution function. First, note that the price paid/received for limit orders depends on the stock price at time $k$. This implies that at $k$, if the agent posts a sell limit order, and the binary random variable $L_k^-$ (which depends on the binary random variable $M_k^+$) is equal to 1, then the agent's order is filled ``between timesteps'' $k$ and $k+1$, but using the price at time $k$. Second, since the second dimension $T(\bz_k, \omega_k)^{(2)} = \Delta_{k+1} = \sgn(s_{k+1} - s_k)$ determines the directionality of the price jump between times $k$ and $k+1$, multiplying it by the random variable $\eta_{k,\bz_k}$ determines the the size of the price change.

\subsection*{Max Terminal Wealth (Discrete)}
Following traditional dynamic programming, we introduce the value function $V_k^{\bu}$. Here, our objective is to maximize the terminal wealth performance criteria given by
\begin{equation}
V_k^{\bu} (x,s,\bz,q) = \E \left[ W_T^{\bu} \right] = \E_{k,x,s,\bz,q}\left[ X_T^{\bu} + Q_T^{\bu}(S_T - \sgn(Q_T^{\bu})\pi) - \alpha {(Q_T^{\bu})}^2 \right]
\end{equation}
where, as before, the notation $\E_{k,x,s,\bz,q}[ \; \cdot \; ]$ represents the conditional expectation
\[ \E [ \; \cdot \; | \; X_k = x, S_k = s, \bZ_k = \bz, Q_k = q] \]
In this case, our dynamic programming equations (DPEs) are given by
\begin{align}
V_k (x,s,\bz,q) & = \sup_{\bu} \bigl\lbrace \E_{\bw} \left[ V_{k+1} (f((x,s,\bz,q),\bu,\bw_k) \right] \bigr\rbrace \\
V_T (x,s,\bz,q) & = \sup_{\bu} \bigl\lbrace \E \left[ x + q(s-\sgn(q)\pi) - \alpha q^2 \right] \bigr\rbrace
\end{align}
where expectation is with respect to the random vector $\bw_k$.

To simplify the DPEs, we introduce a now familiar ansatz:
\begin{equation}
\label{eq:ansatzHcase1discrete}
V_k = x + q(s-\sgn(q)\pi) + h_k(\bz,q)
\end{equation}
with boundary condition $h_k(\bz,0) = 0$ and terminal condition $h_T(\bz,q) = -\alpha q^2$. Substituting this into the DPE, we obtain
\begin{align}
0 & = \sup_{\bu} \bigl\lbrace \E_{\bw} \left[ V_{k+1} (f((x,s,\bz,q),\bu,\bw_k) \right] \bigr\rbrace - V_k (x,s,\bz,q)  \nonumber \\
\begin{split} & = \sup_{\bu} \biggl\lbrace \E_{\bw} \biggl[
(s + \pi + \delta^-)L_k^- - (s - \pi - \delta^+)L_k^+ + (s - \pi)M_k^- - (s + \pi)M_k^+ \\
& \hphantom{\sup_{\bu} \biggl\lbrace \E_{\bw} \biggl[ {}+{}} + (L_k^+ - L_k^- + M_k^+ - M_k^-) \\
& \hphantom{\sup_{\bu} \biggl\lbrace \E_{\bw} \biggl[ {}+{}} \times \bigl( s + \eta_{0,\bz} T(\bz, \omega)^{(2)}  -\sgn( q + L_k^+ - L_k^- + M_k^+ - M_k^-)\pi   \bigr) \\
& \hphantom{\sup_{\bu} \biggl\lbrace \E_{\bw} \biggl[ {}+{}} + q\left( \eta_{0,\bz} T(\bz, \omega)^{(2)}  -\left( \sgn( q + L_k^+ - L_k^- + M_k^+ - M_k^-) - \sgn(q) \right) \pi \right) \\
& \hphantom{\sup_{\bu} \biggl\lbrace \E_{\bw} \biggl[ {}+{}} + h_{k+1}(T(\bz,\omega), q + L_k^+ - L_k^- + M_k^+ - M_k^-) -  h_k(\bz,q) \biggr] \biggr\rbrace 
\end{split}
\end{align}
Since our buy/sell market order controls are binary, the supremum over the control vector $\bu$ can be treated as a simultaneous supremum over $\delta^\pm$ and maximum over the four possible values for $M^\pm$. Notably, however, a quick substitution shows that the case where $M^+ = M^- = 1$ is not possible as it is always strictly $2\pi$ less in value than the case of only limit orders, where $M^+ = M^- = 0$. This should be evident, as buying and selling with market orders in a single timestep yields a guaranteed loss as the agent is forced to cross the spread. Thus, our DPE takes the form:
\begin{equation}
\label{eq:discreteDPE}
\begin{split}
0 & = \max \Biggl\lbrace 
%%% Only Limit Orders
\sup_{\delta^\pm} \biggl\lbrace \E_{\bw} \biggl[
(s + \pi + \delta^-)L_k^- - (s - \pi - \delta^+)L_k^+ \\
& \hphantom{\sup_{\bu} \biggl\lbrace \E_{\bw} \biggl[ {}+{}} + (L_k^+ - L_k^-) \bigl( s + \eta_{0,\bz} T(\bz, \omega)^{(2)}  -\sgn( q + L_k^+ - L_k^-)\pi   \bigr) \\
& \hphantom{\sup_{\bu} \biggl\lbrace \E_{\bw} \biggl[ {}+{}} + q\left( \eta_{0,\bz} T(\bz, \omega)^{(2)}  -\left( \sgn( q + L_k^+ - L_k^-) - \sgn(q) \right) \pi \right) \\
& \hphantom{\sup_{\bu} \biggl\lbrace \E_{\bw} \biggl[ {}+{}} + h_{k+1}(T(\bz,\omega), q + L_k^+ - L_k^- ) -  h_k(\bz,q) \biggr] \biggr\rbrace \; ;\\
%%% Limit Orders + Market Buy
& \hphantom{{}={} \max \biggl\lbrace } \sup_{\delta^\pm} \biggl\lbrace \E_{\bw} \biggl[
(s + \pi + \delta^-)L_k^- - (s - \pi - \delta^+)L_k^+ - (s + \pi)\\
& \hphantom{\sup_{\bu} \biggl\lbrace \E_{\bw} \biggl[ {}+{}} + (L_k^+ - L_k^- +1) \bigl( s + \eta_{0,\bz} T(\bz, \omega)^{(2)}  -\sgn( q + L_k^+ - L_k^- +1)\pi   \bigr) \\
& \hphantom{\sup_{\bu} \biggl\lbrace \E_{\bw} \biggl[ {}+{}} + q\left( \eta_{0,\bz} T(\bz, \omega)^{(2)}  -\left( \sgn( q + L_k^+ - L_k^- +1) - \sgn(q) \right) \pi \right) \\
& \hphantom{\sup_{\bu} \biggl\lbrace \E_{\bw} \biggl[ {}+{}} + h_{k+1}(T(\bz,\omega), q + L_k^+ - L_k^- + 1) -  h_k(\bz,q) \biggr] \biggr\rbrace \; ; \\
%%% Limit Orders + Market Sell
& \hphantom{{}={} \max \biggl\lbrace } \sup_{\delta^\pm} \biggl\lbrace \E_{\bw} \biggl[
(s + \pi + \delta^-)L_k^- - (s - \pi - \delta^+)L_k^+ + (s - \pi) \\
& \hphantom{\sup_{\bu} \biggl\lbrace \E_{\bw} \biggl[ {}+{}} + (L_k^+ - L_k^- -1) \bigl( s + \eta_{0,\bz} T(\bz, \omega)^{(2)}  -\sgn( q + L_k^+ - L_k^- -1)\pi   \bigr) \\
& \hphantom{\sup_{\bu} \biggl\lbrace \E_{\bw} \biggl[ {}+{}} + q\left( \eta_{0,\bz} T(\bz, \omega)^{(2)}  -\left( \sgn( q + L_k^+ - L_k^- -1) - \sgn(q) \right) \pi \right) \\
& \hphantom{\sup_{\bu} \biggl\lbrace \E_{\bw} \biggl[ {}+{}} + h_{k+1}(T(\bz,\omega), q + L_k^+ - L_k^- -1) -  h_k(\bz,q) \biggr] \biggr\rbrace  \Biggr\rbrace
\end{split}
\end{equation}

We'll begin by concentrating on the first case where $M^+ = M^- = 0$. Thus, we want to solve
\begin{equation}
\label{eq:discretesup1}
\begin{split}
& \sup_{\delta^\pm} \biggl\lbrace \E_{\bw} \biggl[
(s + \pi + \delta^-)L_k^- - (s - \pi - \delta^+)L_k^+ \\
& \hphantom{\sup_{\bu} \biggl\lbrace \E_{\bw} \biggl[ {}+{}} + (L_k^+ - L_k^-) \bigl( s + \eta_{0,\bz} T(\bz, \omega)^{(2)}  -\sgn( q + L_k^+ - L_k^-)\pi   \bigr) \\
& \hphantom{\sup_{\bu} \biggl\lbrace \E_{\bw} \biggl[ {}+{}} + q\left( \eta_{0,\bz} T(\bz, \omega)^{(2)}  -\left( \sgn( q + L_k^+ - L_k^-) - \sgn(q) \right) \pi \right) \\
& \hphantom{\sup_{\bu} \biggl\lbrace \E_{\bw} \biggl[ {}+{}} + h_{k+1}(T(\bz,\omega), q + L_k^+ - L_k^- ) -  h_k(\bz,q) \biggr] \biggr\rbrace
\end{split}
\end{equation}
You might recall that things got pretty messy pretty fast. Previously we had set up the problem such that at each timestep $k$ there can be multiple other agents' market orders ($K_k^+$) arriving, and these were Poisson distributed. For each arriving order, the probability of our posted limit order being filled was $e^{-\kappa \delta^-}$. We're going to modify this slightly. Keeping the market orders as Poisson distributed, we have that $\P [ K_k^+ = 0] = \frac{e^{-\lambda (\bz) \Delta t} (\lambda (\bz) \Delta t)^0}{0!} = e^{-\lambda (\bz) \Delta t}$, and so the probability of seeing some positive number of market orders is
\begin{equation}
\label{eq:discretepositiveKplus}
\P [ K_k^+ > 0] = 1 - e^{-\lambda (\bz) \Delta t}
\end{equation}
Now we make the simplified assumption that the \textit{aggregate} of the orders walks the limit order book to a depth of $p_k$, and if $p_k > \delta^-$, then our sell limit order is lifted. Thus we have the following preliminary results:
\begin{align*}
\P [ L_k^- = 1 | K_k^+ > 0] & = e^{-\kappa \delta^-} \\
\P [ L_k^- = 0 | K_k^+ > 0] & = 1 - e^{-\kappa \delta^-} \\
\E [ L_k^- ] & = \P [ L_k^- = 1 | K_k^+ > 0] \cdot \P[K_k^+ > 0] \\
& = ( 1 - e^{-\lambda (\bz) \Delta t} ) e^{-\kappa \delta^-} \\
\end{align*}
For ease of notation, we'll write the probability of the $L_k^- = 1$ event as $p(\delta^-)$. This gives us the additional results:
\begin{align*}
\P [ L_k^- = 1] & = p(\delta^-) = \E [ L_k^-] \\
\P [ L_k^- = 0] & = 1 - p(\delta^-) \\
\partial_{\delta^-} \P [ L_k^- = 1]  & = -\kappa p(\delta^-) \\
\partial_{\delta^-} \P [ L_k^- = 0] & = \kappa p(\delta^-) \\
\end{align*}
Contributing to last week's frustration, a la our man Malcolm X, was an error in calculation. I had mistakenly calculated $\E [ L_k^- \sgn(q + L_k^+ - L_k^-)] = \E [ L_k^-] \E[\sgn(q + L_k^+ - L_k^-)]$, as if to say they are independent, but of course they are not. This is fixed below. On the contrary, $\E [ L_k^- \eta_{0,\bz}T(\bz,\omega)^{(2)}]$ was correctly treated as the product of independent random variables. We have that $\E[L_k^-] = ( 1 - e^{-\lambda (\bz) \Delta t} ) e^{-\kappa \delta^-}$ is clearly dependent on $\bz$, but these expectations are over the vector of random variables $\bw = (K^+, K^-, \omega)$ and are evaluated at a given point $\bz$.

Let's pre-compute some of the terms that we'll encounter in the supremum, namely the expectations of the random variables. 
\begin{align}
\E [\sgn(q + L_k^+ - L_k^-)] & = \P[L_k^- = 1] \cdot \P[L_k^+ = 1] \cdot \sgn(q) \nonumber \\
& \quad + \P[L_k^- = 1] \cdot \P[L_k^+ = 0] \cdot \sgn(q - 1) \nonumber \\
& \quad +  \P[L_k^- = 0] \cdot \P[L_k^+ = 1] \cdot \sgn(q+1) \nonumber \\
& \quad + \P[L_k^- = 0] \cdot \P[L_k^+ = 0] \cdot \sgn(q) \nonumber \\
& = p(\delta^-)p(\delta^+) \sgn(q) \nonumber \\
& \quad + p(\delta^-) (1-p(\delta^+)) \sgn(q - 1) \nonumber \\
& \quad + (1 - p(\delta^-)) p(\delta^+)  \sgn(q+1) \nonumber \\
& \quad + (1 - p(\delta^-)) (1-p(\delta^+))  \sgn(q) \nonumber \\
& = \sgn(q) \bigl[ 1 - p(\delta^+) - p(\delta^-) + 2 p(\delta^+) p(\delta^-) \bigr] \nonumber \\
& \quad + \sgn(q-1) \bigl[ p(\delta^-)  - p(\delta^+) p(\delta^-) \bigr]  \nonumber \\
& \quad + \sgn(q+1) \bigl[ p(\delta^+)  - p(\delta^+) p(\delta^-) \bigr]  \nonumber \\
& = \begin{cases} 
1 & q \geq 2 \\
1 - p(\delta^-)(1 - p(\delta^+)) & q = 1 \\
p(\delta^+) - p(\delta^-) & q = 0 \\
-\bigl[ 1 - p(\delta^+)(1 - p(\delta^-)) \bigr] & q = -1 \\
-1 & q \leq -2
\end{cases} \\
& = \Phi(q, \delta^+, \delta^-)
\end{align}
Similarly:
\begin{align}
\E [ L_k^+ \sgn(q + L_k^+ - L_k^-)] & = \P[L_k^- = 1] \cdot \P[L_k^+ = 1] \cdot \sgn(q) \nonumber \\
& \quad + \P[L_k^- = 1] \cdot \P[L_k^+ = 0] \cdot 0 \sgn(q - 1) \nonumber \\
& \quad +  \P[L_k^- = 0] \cdot \P[L_k^+ = 1] \cdot \sgn(q+1) \nonumber \\
& \quad + \P[L_k^- = 0] \cdot \P[L_k^+ = 0] \cdot 0 \sgn(q) \nonumber \\
& = p(\delta^+) \bigl[ p(\delta^-) \sgn(q) + (1-p(\delta^-) \sgn(q+1) \bigr] \nonumber \\
& = p(\delta^+) \begin{cases} 
1 & q \geq 2 \\
1 & q = 1 \\
(1 - p(\delta^-)) & q = 0 \\
-p(\delta^-) & q = -1 \\
-1 & q \leq -2
\end{cases} \\
& = p(\delta^+) \Psi(q, \delta^-)
\end{align}
and
\begin{align}
\E [ L_k^- \sgn(q + L_k^+ - L_k^-)] & = p(\delta^-) \bigl[ p(\delta^+) \sgn(q) + (1-p(\delta^+)) \sgn(q-1) \bigr] \nonumber \\
& = p(\delta^-) \begin{cases} 
1 & q \geq 2 \\
p(\delta^+) & q = 1 \\
-(1 - p(\delta^+)) & q = 0 \\
-1 & q = -1 \\
-1 & q \leq -2
\end{cases} \\
& = p(\delta^-) \Upsilon(q, \delta^+)
\end{align}
We'll also require the partial derivatives of these expectations, which we can easily compute. Below we'll use the simplified notation $\Phi_+$ to denote the function closely associated with the partial derivative of $ \Phi$ with respect to $\delta^+$.
\begin{align}
\partial_{\delta^-} \E [\sgn(q + L_k^+ - L_k^-)] = \partial_{\delta^-} \Phi(q, \delta^+, \delta^-) & = \kappa p(\delta^-) \begin{cases} 
0 & q \geq 2 \\
(1-p(\delta^+)) & q = 1 \\
1 & q = 0 \\
p(\delta^+)  & q = -1 \\
0 & q \leq -2
\end{cases} \\
& = \kappa p(\delta^-) \Phi_-(q,\delta^+) \\
\partial_{\delta^+} \E [\sgn(q + L_k^+ - L_k^-)] =\partial_{\delta^+} \Phi(q, \delta^+, \delta^-) & =  \kappa p(\delta^+) \begin{cases} 
0 & q \geq 2 \\
- p(\delta^-) & q = 1 \\
- 1 & q = 0 \\
- (1 - p(\delta^-)) & q = -1 \\
0 & q \leq -2
\end{cases} \\
& = \kappa p(\delta^+)\Phi_+(q,\delta^-) \\
\partial_{\delta^-} \E [L_k^+ \sgn(q + L_k^+ - L_k^-)] = \partial_{\delta^-} p(\delta^+) \Psi(q, \delta^-) & = \kappa p(\delta^+) p(\delta^-)\begin{cases} 
0 & q \geq 2 \\
0 & q = 1 \\
1 & q = 0 \\
1 & q = -1 \\
0 & q \leq -2
\end{cases} \\
& = \kappa p(\delta^+) p(\delta^-) \Psi_-(q) \\
\partial_{\delta^+} \E [L_k^+ \sgn(q + L_k^+ - L_k^-)] = \partial_{\delta^+} p(\delta^+) \Psi(q, \delta^-) & = -\kappa p(\delta^+) \Psi(q, \delta^-)
 \\
\partial_{\delta^-} \E [L_k^- \sgn(q + L_k^+ - L_k^-)] = \partial_{\delta^-} p(\delta^-) \Upsilon(q, \delta^+) & = -\kappa p(\delta^-) \Upsilon(q, \delta^+)
 \\
 \partial_{\delta^+} \E [L_k^- \sgn(q + L_k^+ - L_k^-)] = \partial_{\delta^+} p(\delta^-) \Upsilon(q, \delta^+)& = \kappa p(\delta^+) p(\delta^-) \begin{cases} 
0 & q \geq 2 \\
-1 & q = 1 \\
-1 & q = 0 \\
0 & q = -1 \\
0 & q \leq -2
\end{cases} \\
& = \kappa p(\delta^+) p(\delta^-)\Upsilon_+(q)
\end{align}
Recalling that we have  $\bP$ the transition matrix for the Markov Chain $\bZ$, with $\bP_{\bz, \bj} = \P[\bZ_{k+1} = \bj | \bZ_k = \bz]$, then we can also write:
\begin{equation}
\begin{split}
\E[h_{k+1}(T(\bz,\omega), q + L_k^+ - L_k^- )] & = \sum_\bj \bP_{\bz,\bj} \biggl[ h_{k+1}(\bj, q) \bigl[ 1 - p(\delta^+) - p(\delta^-) + 2 p(\delta^+) p(\delta^-) \bigr]  \\
& \hphantom{\sum_\bj \bP_{\bz,\bj} \biggl[ {}+{}} + h_{k+1}(\bj, q-1) \bigl[ p(\delta^-)  - p(\delta^+) p(\delta^-) \bigr]   \\
& \hphantom{\sum_\bj \bP_{\bz,\bj} \biggl[ {}+{}} + h_{k+1}(\bj, q+1) \bigl[ p(\delta^+)  - p(\delta^+) p(\delta^-) \bigr] \biggr]
\end{split}
\end{equation}
and its partial derivatives as
\begin{align}
\begin{split}
\partial_{\delta^-} \E[h_{k+1}(T(\bz,\omega), q + L_k^+ - L_k^- )] & = 
\sum_\bj \bP_{\bz,\bj} \biggl[ h_{k+1}(\bj, q) \bigl[ \kappa p(\delta^-) - 2 \kappa p(\delta^+) p(\delta^-) \bigr]  \\
& \hphantom{\sum_\bj \bP_{\bz,\bj} \biggl[ {}+{}} + h_{k+1}(\bj, q-1) \bigl[ -\kappa p(\delta^-) + \kappa p(\delta^+) p(\delta^-) \bigr]   \\
& \hphantom{\sum_\bj \bP_{\bz,\bj} \biggl[ {}+{}} + h_{k+1}(\bj, q+1) \bigl[ \kappa p(\delta^+) p(\delta^-) \bigr] \biggr]
\end{split} \\
\begin{split}
& = \kappa p(\delta^-) \sum_\bj \bP_{\bz,\bj} \biggl[ h_{k+1}(\bj, q) \bigl[ 1 - 2 p(\delta^+) \bigr]  \\
& \hphantom{d(\delta^-) \sum_\bj \bP_{\bz,\bj} \biggl[ {}+{}} + h_{k+1}(\bj, q-1) \bigl[-1 + p(\delta^+) \bigr]   \\
& \hphantom{d(\delta^-) \sum_\bj \bP_{\bz,\bj} \biggl[ {}+{}} + h_{k+1}(\bj, q+1) \bigl[ p(\delta^+) \bigr] \biggr]
\end{split} \\
\begin{split}
\partial_{\delta^+} \E[h_{k+1}(T(\bz,\omega), q + L_k^+ - L_k^- )] & = 
\kappa p(\delta^+) \sum_\bj \bP_{\bz,\bj} \biggl[ h_{k+1}(\bj, q) \bigl[ 1 - 2 p(\delta^-) \bigr]  \\
& \hphantom{d(\delta^-) \sum_\bj \bP_{\bz,\bj} \biggl[ {}+{}} + h_{k+1}(\bj, q-1) \bigl[ p(\delta^-) \bigr]   \\
& \hphantom{d(\delta^-) \sum_\bj \bP_{\bz,\bj} \biggl[ {}+{}} + h_{k+1}(\bj, q+1) \bigl[ -1 + p(\delta^-) \bigr] \biggr]
\end{split}
\end{align}

Now we tackle solving the supremum in equation $\ref{eq:discretesup1}$. First we consider the first-order condition on $\delta^-$, namely that the partial derivative with respect to it must be equal to zero.
\begin{align}
\begin{split}
0 & = \partial_{\delta^-} \biggl\lbrace 
(s + \pi + \delta^-)\E [L_k^-] - (s - \pi - \delta^+)\E[L_k^+] \\
& \hphantom{\partial_{\delta^-} \biggl\lbrace {}+{}} + \E[L_k^+] \left( s + \E[\eta_{0,\bz} T(\bz, \omega)^{(2)}] \right)  - \pi \E \left[L_k^+ \sgn( q + L_k^+ - L_k^-) \right] \\
& \hphantom{\partial_{\delta^-} \biggl\lbrace {}+{}} - \E[L_k^-] \left( s + \E[\eta_{0,\bz} T(\bz, \omega)^{(2)}] \right) + \pi \E \left[ L_k^- \sgn( q + L_k^+ - L_k^-) \right] \\
& \hphantom{\partial_{\delta^-} \biggl\lbrace {}+{}} + q \E[ \eta_{0,\bz} T(\bz, \omega)^{(2)}]  - q \pi \E[ \sgn( q + L_k^+ - L_k^-)] + q \pi \sgn(q)  \\
& \hphantom{\partial_{\delta^-} \biggl\lbrace {}+{}} + \E \left[ h_{k+1}(T(\bz,\omega), q + L_k^+ - L_k^- ) \right] -  h_k(\bz,q) \biggr\rbrace
\end{split} \\
\begin{split}
& = \partial_{\delta^-} \biggl\lbrace 
(s + \pi + \delta^-)\E [L_k^-] - \pi \E \left[L_k^+ \sgn( q + L_k^+ - L_k^-) \right] \\
& \hphantom{\partial_{\delta^-} \biggl\lbrace {}+{}} - \E[L_k^-] \left( s + \E[\eta_{0,\bz} T(\bz, \omega)^{(2)}] \right) + \pi \E \left[ L_k^- \sgn( q + L_k^+ - L_k^-) \right] \\
& \hphantom{\partial_{\delta^-} \biggl\lbrace {}+{}} - q \pi \E[ \sgn( q + L_k^+ - L_k^-)]  + \E \left[ h_{k+1}(T(\bz,\omega), q + L_k^+ - L_k^- ) \right]  \biggr\rbrace
\end{split} \\
\begin{split}
& = p(\delta^-) - \kappa p(\delta^-) (s + \pi + \delta^-) - \pi \kappa p(\delta^+)p(\delta^-)\Psi_-(q) \\
& \hphantom{{}={}} + \kappa p(\delta^-) \left( s + \E[\eta_{0,\bz} T(\bz, \omega)^{(2)}] \right) - \pi \kappa p(\delta^-) \Upsilon(q,\delta^+) - q \pi \kappa p(\delta^-) \Phi_-(q,\delta^+) \\
& \hphantom{{}={}} + \kappa p(\delta^-) \sum_\bj \bP_{\bz,\bj} \biggl[ h_{k+1}(\bj, q) \bigl[ 1 - 2 p(\delta^+) \bigr] + h_{k+1}(\bj, q-1) \bigl[-1 + p(\delta^+) \bigr] \\
& \hphantom{{}={} + \kappa p(\delta^-) \sum_\bj \bP_{\bz,\bj} \biggl[} + h_{k+1}(\bj, q+1) \bigl[ p(\delta^+) \bigr] \biggr]
\end{split}
\end{align}
Dividing through by $\kappa p(\delta^-)$, which is nonzero, and re-arranging, we find that the optimal sell posting depth is given by
\begin{align}
\begin{split}
{\delta^-}^* & = \frac{1}{\kappa} + \E[\eta_{0,\bz} T(\bz, \omega)^{(2)}] - \pi \left( 1 + p(\delta^+)\Psi_-(q) + \Upsilon(q,\delta^+) + q \Phi_-(q,\delta^+) \right) \\
& \hphantom{{}={}} +  \sum_\bj \bP_{\bz,\bj} \biggl[ h_{k+1}(\bj, q) \bigl[ 1 - 2 p(\delta^+) \bigr] + h_{k+1}(\bj, q-1) \bigl[-1 + p(\delta^+) \bigr] + h_{k+1}(\bj, q+1) \bigl[ p(\delta^+) \bigr] \biggr]
\end{split} \\
\begin{split}
& = \frac{1}{\kappa} + \E[\eta_{0,\bz} T(\bz, \omega)^{(2)}] - 2 \pi \left( \indicator_{q \geq 1} + p(\delta^+)\indicator_{q = 0} \right) \\
& \hphantom{{}={}} +  \sum_\bj \bP_{\bz,\bj} \biggl[ h_{k+1}(\bj, q) \bigl[ 1 - 2 p(\delta^+) \bigr] + h_{k+1}(\bj, q-1) \bigl[-1 + p(\delta^+) \bigr] + h_{k+1}(\bj, q+1) \bigl[ p(\delta^+) \bigr] \biggr]
\end{split} \\
\begin{split}
\label{eq:discretedeltaminusreferential}
{\delta^-}^* & = \frac{1}{\kappa} + \E[\eta_{0,\bz} T(\bz, \omega)^{(2)}] - 2 \pi \indicator_{q \geq 1} + \sum_\bj \bP_{\bz,\bj} \biggl[ h_{k+1}(\bj, q) - h_{k+1}(\bj, q-1) \biggr] \\
& \hphantom{{}={}} -p(\delta^+) \left( 2 \pi \indicator_{q = 0} - \sum_\bj \bP_{\bz,\bj} \biggl[ h_{k+1}(\bj, q-1) + h_{k+1}(\bj, q+1) -2 h_{k+1}(\bj, q)  \biggr] \right)
\end{split}
\intertext{And similarly, for the optimal buy posting depth:}
\begin{split}
\label{eq:discretedeltaplusreferential}
{\delta^+}^* & = \frac{1}{\kappa} - \E[\eta_{0,\bz} T(\bz, \omega)^{(2)}] - 2 \pi \indicator_{q \leq -1} + \sum_\bj \bP_{\bz,\bj} \biggl[ h_{k+1}(\bj, q) - h_{k+1}(\bj, q+1) \biggr] \\
& \hphantom{{}={}} -p(\delta^-) \left( 2 \pi \indicator_{q = 0} - \sum_\bj \bP_{\bz,\bj} \biggl[ h_{k+1}(\bj, q-1) + h_{k+1}(\bj, q+1) -2 h_{k+1}(\bj, q)  \biggr] \right)
\end{split}
\end{align}
For ease of notation we'll write $\aleph = \sum_\bj \bP_{\bz,\bj} \left[ h_{k+1}(\bj, q-1) + h_{k+1}(\bj, q+1) -2 h_{k+1}(\bj, q)  \right]$. Now, assuming we behave optimally on both the buy and sell sides simultaneously, we can substitute equation \ref{eq:discretedeltaplusreferential} into equation \ref{eq:discretedeltaminusreferential}, and vice versa, while evaluating both at ${\delta^+}^*$ and ${\delta^-}^*$ to obtain the optimal posting depths in feedback form:
\begin{equation}
\label{eq:discretedeltaminusfeedback}
\boxed{
\begin{split}
{\delta^-}^* & = \frac{1}{\kappa} + \E[\eta_{0,\bz} T(\bz, \omega)^{(2)}] - 2 \pi \indicator_{q \geq 1} + \sum_\bj \bP_{\bz,\bj} \biggl[ h_{k+1}(\bj, q) - h_{k+1}(\bj, q-1) \biggr] \\
& \hphantom{{}={}} - (1-e^{\lambda (\bz) \Delta t})e^{-\kappa \bigl( \frac{1}{\kappa} - \E[\eta_{0,\bz} T(\bz, \omega)^{(2)}] - 2 \pi \indicator_{q \leq -1} + \sum_\bj \bP_{\bz,\bj} \bigl[ h_{k+1}(\bj, q) - h_{k+1}(\bj, q+1) \bigr] \bigr) } \\
& \hphantom{{}={} {}-{}} \times e^{\kappa (1-e^{\lambda (\bz) \Delta t})e^{-\kappa {\delta^-}^*} \left( 2 \pi \indicator_{q = 0} - \aleph \right)} \left( 2 \pi \indicator_{q = 0} - \aleph \right)
\end{split}
}
\end{equation}
\begin{equation}
\label{eq:discretedeltaplusfeedback}
\boxed{
\begin{split}
{\delta^+}^* & = \frac{1}{\kappa} - \E[\eta_{0,\bz} T(\bz, \omega)^{(2)}] - 2 \pi \indicator_{q \leq -1} + \sum_\bj \bP_{\bz,\bj} \biggl[ h_{k+1}(\bj, q) - h_{k+1}(\bj, q+1) \biggr] \\
& \hphantom{{}={}} - (1-e^{\lambda (\bz) \Delta t})e^{-\kappa \bigl( \frac{1}{\kappa} + \E[\eta_{0,\bz} T(\bz, \omega)^{(2)}] - 2 \pi \indicator_{q \geq 1} + \sum_\bj \bP_{\bz,\bj} \bigl[ h_{k+1}(\bj, q) - h_{k+1}(\bj, q-1) \bigr] \bigr) } \\
& \hphantom{{}={} {}-{}} \times e^{\kappa (1-e^{\lambda (\bz) \Delta t})e^{-\kappa {\delta^+}^*} \left( 2 \pi \indicator_{q = 0} - \aleph \right)} \left( 2 \pi \indicator_{q = 0} - \aleph \right)
\end{split}
}
\end{equation}
\subsection*{DPE Cases 2 and 3}
We have derived the optimal LO posting depths for the first case of the DPE, where no market orders are placed. The analysis of the other two cases, where $M_k^+ = 1$ and $M_k^- = 1$, respectively, proceeds almost identically. Looking first at the case $M_k^+ = 1$, we try to solve:
\begin{equation}
\label{eq:discretesup2}
\begin{split}
& \sup_{\delta^\pm} \biggl\lbrace \E_{\bw} \biggl[
(s + \pi + \delta^-)L_k^- - (s - \pi - \delta^+)L_k^+ - (s + \pi)\\
& \hphantom{\sup_{\bu} \biggl\lbrace \E_{\bw} \biggl[ {}+{}} + (L_k^+ - L_k^- +1) \bigl( s + \eta_{0,\bz} T(\bz, \omega)^{(2)}  -\sgn( q + L_k^+ - L_k^- +1)\pi   \bigr) \\
& \hphantom{\sup_{\bu} \biggl\lbrace \E_{\bw} \biggl[ {}+{}} + q\left( \eta_{0,\bz} T(\bz, \omega)^{(2)}  -\left( \sgn( q + L_k^+ - L_k^- +1) - \sgn(q) \right) \pi \right) \\
& \hphantom{\sup_{\bu} \biggl\lbrace \E_{\bw} \biggl[ {}+{}} + h_{k+1}(T(\bz,\omega), q + L_k^+ - L_k^- + 1) -  h_k(\bz,q) \biggr] \biggr\rbrace \; ; \\
\end{split}
\end{equation}
We find that the functions $\Phi, \Psi, \Upsilon$ are identical but evaluated at $q+1$. Thus, for example, we have:
\begin{equation}
\label{eq:discretephioffset}
\Phi(q+1, \delta^+, \delta^-) = \begin{cases} 
1 & q \geq 1 \\
1 - p(\delta^-)(1 - p(\delta^+)) & q = 0 \\
p(\delta^+) - p(\delta^-) & q = -1 \\
-\bigl[ 1 - p(\delta^+)(1 - p(\delta^-)) \bigr] & q = -2 \\
-1 & q \leq -3
\end{cases}
\end{equation}
In a similar fashion to the previous analysis, we obtain the posting depths for the second case (indicated with a subscript 2)
\begin{align}
\label{eq:discretedeltaminus2referential}
\begin{split}
{\delta_{2}^{-}}^* & = \frac{1}{\kappa} + \E[\eta_{0,\bz} T(\bz, \omega)^{(2)}] - 2 \pi \indicator_{q \geq 0} + \sum_\bj \bP_{\bz,\bj} \biggl[ h_{k+1}(\bj, q+1) - h_{k+1}(\bj, q) \biggr] \\
& \hphantom{{}={}} -p(\delta^+) \left( 2 \pi \indicator_{q = -1} - \sum_\bj \bP_{\bz,\bj} \biggl[ h_{k+1}(\bj, q) + h_{k+1}(\bj, q+2) -2 h_{k+1}(\bj, q+1)  \biggr] \right)
\end{split} \\
\begin{split}
\label{eq:discretedeltaplus2referential}
{\delta_{2}^+}^* & = \frac{1}{\kappa} - \E[\eta_{0,\bz} T(\bz, \omega)^{(2)}] - 2 \pi \indicator_{q \leq -2} + \sum_\bj \bP_{\bz,\bj} \biggl[ h_{k+1}(\bj, q+1) - h_{k+1}(\bj, q+2) \biggr] \\
& \hphantom{{}={}} -p(\delta^-) \left( 2 \pi \indicator_{q = -1} - \sum_\bj \bP_{\bz,\bj} \biggl[ h_{k+1}(\bj, q) + h_{k+1}(\bj, q+2) -2 h_{k+1}(\bj, q+1)  \biggr] \right)
\end{split}
\end{align}
Again, substitution equations \ref{eq:discretedeltaminus2referential} and \ref{eq:discretedeltaplus2referential} into one another and evaluating both at the optimal posting depths, we obtain
\begin{equation}
\label{eq:discretedeltaminus2feedback}
\boxed{
\begin{split}
{\delta_{2}^-}^* & = \frac{1}{\kappa} + \E[\eta_{0,\bz} T(\bz, \omega)^{(2)}] - 2 \pi \indicator_{q \geq 0} + \sum_\bj \bP_{\bz,\bj} \biggl[ h_{k+1}(\bj, q+1) - h_{k+1}(\bj, q) \biggr] \\
& \hphantom{{}={}} - (1-e^{\lambda (\bz) \Delta t})e^{-\kappa \bigl( \frac{1}{\kappa} - \E[\eta_{0,\bz} T(\bz, \omega)^{(2)}] - 2 \pi \indicator_{q \leq -2} + \sum_\bj \bP_{\bz,\bj} \bigl[ h_{k+1}(\bj, q+1) - h_{k+1}(\bj, q+2) \bigr] \bigr) } \\
& \hphantom{{}={} {}-{}} \times e^{\kappa (1-e^{\lambda (\bz) \Delta t})e^{-\kappa {\delta_{2}^-}^*} \left( 2 \pi \indicator_{q = -1} - \aleph_{(+1)} \right)} \left( 2 \pi \indicator_{q = -1} - \aleph_{(+1)} \right)
\end{split}
}
\end{equation}
\begin{equation}
\label{eq:discretedeltaplus2feedback}
\boxed{
\begin{split}
{\delta_{2}^+}^* & = \frac{1}{\kappa} - \E[\eta_{0,\bz} T(\bz, \omega)^{(2)}] - 2 \pi \indicator_{q \leq -2} + \sum_\bj \bP_{\bz,\bj} \biggl[ h_{k+1}(\bj, q+1) - h_{k+1}(\bj, q+2) \biggr] \\
& \hphantom{{}={}} - (1-e^{\lambda (\bz) \Delta t})e^{-\kappa \bigl( \frac{1}{\kappa} + \E[\eta_{0,\bz} T(\bz, \omega)^{(2)}] - 2 \pi \indicator_{q \geq 0} + \sum_\bj \bP_{\bz,\bj} \bigl[ h_{k+1}(\bj, q+1) - h_{k+1}(\bj, q) \bigr] \bigr) } \\
& \hphantom{{}={} {}-{}} \times e^{\kappa (1-e^{\lambda (\bz) \Delta t})e^{-\kappa {\delta_{2}^+}^*} \left( 2 \pi \indicator_{q = -1} - \aleph_{(+1)} \right)} \left( 2 \pi \indicator_{q = -1} - \aleph_{(+1)} \right)
\end{split}
}
\end{equation}
Likewise, in the third DPE case where $M_k^+ =0, M_k^-=1$, we get
\begin{equation}
\label{eq:discretedeltaminus3feedback}
\boxed{
\begin{split}
{\delta_{3}^-}^* & = \frac{1}{\kappa} + \E[\eta_{0,\bz} T(\bz, \omega)^{(2)}] - 2 \pi \indicator_{q \geq 2} + \sum_\bj \bP_{\bz,\bj} \biggl[ h_{k+1}(\bj, q-1) - h_{k+1}(\bj, q-2) \biggr] \\
& \hphantom{{}={}} - (1-e^{\lambda (\bz) \Delta t})e^{-\kappa \bigl( \frac{1}{\kappa} - \E[\eta_{0,\bz} T(\bz, \omega)^{(2)}] - 2 \pi \indicator_{q \leq 0} + \sum_\bj \bP_{\bz,\bj} \bigl[ h_{k+1}(\bj, q-1) - h_{k+1}(\bj, q) \bigr] \bigr) } \\
& \hphantom{{}={} {}-{}} \times e^{\kappa (1-e^{\lambda (\bz) \Delta t})e^{-\kappa {\delta_{3}^-}^*} \left( 2 \pi \indicator_{q = 1} - \aleph_{(-1)} \right)} \left( 2 \pi \indicator_{q = 1} - \aleph_{(-1)} \right)
\end{split}
}
\end{equation}
\begin{equation}
\label{eq:discretedeltaplus3feedback}
\boxed{
\begin{split}
{\delta_{3}^+}^* & = \frac{1}{\kappa} - \E[\eta_{0,\bz} T(\bz, \omega)^{(2)}] - 2 \pi \indicator_{q \leq 0} + \sum_\bj \bP_{\bz,\bj} \biggl[ h_{k+1}(\bj, q-1) - h_{k+1}(\bj, q) \biggr] \\
& \hphantom{{}={}} - (1-e^{\lambda (\bz) \Delta t})e^{-\kappa \bigl( \frac{1}{\kappa} + \E[\eta_{0,\bz} T(\bz, \omega)^{(2)}] - 2 \pi \indicator_{q \geq 2} + \sum_\bj \bP_{\bz,\bj} \bigl[ h_{k+1}(\bj, q-1) - h_{k+1}(\bj, q-2) \bigr] \bigr) } \\
& \hphantom{{}={} {}-{}} \times e^{\kappa (1-e^{\lambda (\bz) \Delta t})e^{-\kappa {\delta_{3}^+}^*} \left( 2 \pi \indicator_{q = 1} - \aleph_{(-1)} \right)} \left( 2 \pi \indicator_{q = 1} - \aleph_{(-1)} \right)
\end{split}
}
\end{equation}
The boxed equations for optimal depth in feedback form will need to be solved numerically due to the difficulty in isolating ${\delta^{\pm}}^*$ on one side of the equality. 
\subsection*{Simplifying the DPE}
For reference, we repeat here the DPE we are attempting to solve:
\begin{equation}
\begin{split}
0 & = \max \Biggl\lbrace 
%%% Only Limit Orders
\sup_{\delta^\pm} \biggl\lbrace \E_{\bw} \biggl[
(s + \pi + \delta^-)L_k^- - (s - \pi - \delta^+)L_k^+ \\
& \hphantom{\sup_{\bu} \biggl\lbrace \E_{\bw} \biggl[ {}+{}} + (L_k^+ - L_k^-) \bigl( s + \eta_{0,\bz} T(\bz, \omega)^{(2)}  -\sgn( q + L_k^+ - L_k^-)\pi   \bigr) \\
& \hphantom{\sup_{\bu} \biggl\lbrace \E_{\bw} \biggl[ {}+{}} + q\left( \eta_{0,\bz} T(\bz, \omega)^{(2)}  -\left( \sgn( q + L_k^+ - L_k^-) - \sgn(q) \right) \pi \right) \\
& \hphantom{\sup_{\bu} \biggl\lbrace \E_{\bw} \biggl[ {}+{}} + h_{k+1}(T(\bz,\omega), q + L_k^+ - L_k^- ) -  h_k(\bz,q) \biggr] \biggr\rbrace \; ;\\
%%% Limit Orders + Market Buy
& \hphantom{{}={} \max \biggl\lbrace } \sup_{\delta^\pm} \biggl\lbrace \E_{\bw} \biggl[
(s + \pi + \delta^-)L_k^- - (s - \pi - \delta^+)L_k^+ - (s + \pi)\\
& \hphantom{\sup_{\bu} \biggl\lbrace \E_{\bw} \biggl[ {}+{}} + (L_k^+ - L_k^- +1) \bigl( s + \eta_{0,\bz} T(\bz, \omega)^{(2)}  -\sgn( q + L_k^+ - L_k^- +1)\pi   \bigr) \\
& \hphantom{\sup_{\bu} \biggl\lbrace \E_{\bw} \biggl[ {}+{}} + q\left( \eta_{0,\bz} T(\bz, \omega)^{(2)}  -\left( \sgn( q + L_k^+ - L_k^- +1) - \sgn(q) \right) \pi \right) \\
& \hphantom{\sup_{\bu} \biggl\lbrace \E_{\bw} \biggl[ {}+{}} + h_{k+1}(T(\bz,\omega), q + L_k^+ - L_k^- + 1) -  h_k(\bz,q) \biggr] \biggr\rbrace \; ; \\
%%% Limit Orders + Market Sell
& \hphantom{{}={} \max \biggl\lbrace } \sup_{\delta^\pm} \biggl\lbrace \E_{\bw} \biggl[
(s + \pi + \delta^-)L_k^- - (s - \pi - \delta^+)L_k^+ + (s - \pi) \\
& \hphantom{\sup_{\bu} \biggl\lbrace \E_{\bw} \biggl[ {}+{}} + (L_k^+ - L_k^- -1) \bigl( s + \eta_{0,\bz} T(\bz, \omega)^{(2)}  -\sgn( q + L_k^+ - L_k^- -1)\pi   \bigr) \\
& \hphantom{\sup_{\bu} \biggl\lbrace \E_{\bw} \biggl[ {}+{}} + q\left( \eta_{0,\bz} T(\bz, \omega)^{(2)}  -\left( \sgn( q + L_k^+ - L_k^- -1) - \sgn(q) \right) \pi \right) \\
& \hphantom{\sup_{\bu} \biggl\lbrace \E_{\bw} \biggl[ {}+{}} + h_{k+1}(T(\bz,\omega), q + L_k^+ - L_k^- -1) -  h_k(\bz,q) \biggr] \biggr\rbrace  \Biggr\rbrace
\end{split}
\end{equation}
We now turn to simplifying our DPE by substituting in the optimal posting depths as written in recursive form (e.g. \ref{eq:discretedeltaplusreferential} and \ref{eq:discretedeltaminusreferential}). In doing so we see a incredible amount of cancellation and simplification, and we obtain the rather elegant, and surprisingly simple form of the DPE:
\begin{equation}
\label{eq:discreteDPE}
\begin{split}
h_k(\bz,q) & = \max \Biggl\lbrace 
%%% Only Limit Orders
q\E[\eta_{0,\bz} T(\bz, \omega)^{(2)}] + \frac{1}{\kappa}(p({\delta^+}^*)+p({\delta^-}^*)) -2\pi p({\delta^+}^*)p({\delta^-}^*) \indicator_{q=0} \\ 
& \hphantom{{}={} \max \biggl\lbrace {}+{}} + p({\delta^+}^*)p({\delta^-}^*)\sum_\bj \bP_{\bz,\bj} \left[ h_{k+1}(\bj, q-1) + h_{k+1}(\bj, q+1) -2 h_{k+1}(\bj, q)  \right] \\
& \hphantom{{}={} \max \biggl\lbrace {}+{}} + \sum_{\bj} \bP_{\bz,\bj} h_{k+1}(\bj,q)  \; ;\\
%%% Limit Orders + Market Buy
& \hphantom{{}={} \max \biggl\lbrace} (q+1)\E[\eta_{0,\bz} T(\bz, \omega)^{(2)}] + \frac{1}{\kappa}(p({\delta_{2}^+}^*)+p({\delta_{2}^-}^*)) -2\pi p({\delta_{2}^+}^*)p({\delta_{2}^-}^*) \indicator_{q=-1} \\ 
& \hphantom{{}={} \max \biggl\lbrace {}+{}} + p({\delta_{2}^+}^*)p({\delta_{2}^-}^*)\sum_\bj \bP_{\bz,\bj} \left[ h_{k+1}(\bj, q) + h_{k+1}(\bj, q+2) -2 h_{k+1}(\bj, q+1)  \right] \\
& \hphantom{{}={} \max \biggl\lbrace {}+{}} + \sum_{\bj} \bP_{\bz,\bj} h_{k+1}(\bj,q+1) \; ;\\
%%% Limit Orders + Market Sell
& \hphantom{{}={} \max \biggl\lbrace} (q-1)\E[\eta_{0,\bz} T(\bz, \omega)^{(2)}] + \frac{1}{\kappa}(p({\delta_3^+}^*)+p({\delta_3^-}^*)) -2\pi p({\delta_3^+}^*)p({\delta_3^-}^*) \indicator_{q=1} \\ 
& \hphantom{{}={} \max \biggl\lbrace {}+{}} + p({\delta_3^+}^*)p({\delta_3^-}^*)\sum_\bj \bP_{\bz,\bj} \left[ h_{k+1}(\bj, q-2) + h_{k+1}(\bj, q) -2 h_{k+1}(\bj, q-1)  \right] \\
& \hphantom{{}={} \max \biggl\lbrace {}+{}} + \sum_{\bj} \bP_{\bz,\bj} h_{k+1}(\bj,q-1)  \Biggr\rbrace
\end{split}
\end{equation}
TODO: commentary on this final form.

We now have an explicit means of numerically solving for the optimal posting depths. Since we know the function $h$ at the terminal timestep $T$, we can take one step back to $T-1$ and solve for each of the optimal posting depths in each of the cases. With these values we are then able to calculate the value function $h_{T-1}$, and in doing so determine whether to execute market orders in addition to posting limit orders (essentially taking the $\argmax$ instead of the $\max$ in equation \ref{eq:discreteDPE}). This process then repeats for each step backward.
\end{document}
