% LaTeX set-up adapted from a template by Alan T. Sherman (9/9/98)
%%%%%%%%%%%%%%%%%%%%%%%%%%%%%%%%%%%%%%%%%%%%%%%%%%%%%%%%%%%%%%%%%%%%%%
\documentclass[12pt]{article}
\input{/home/anton/Documents/latex/LaTeXHeader.tex} % put hard path to header.
\usepackage{soul}
\usepackage{hyperref}
\allowdisplaybreaks

%%%%%% Begin document with header and title %%%%%%%%%%%%%%%%%%%%%%%%%

\begin{document}
\mymascheader
\pagestyle{plain}
{\begin{center} {\large {\bf Statistical Arbitrage with \\  Limit Order Book Imbalance}} \end{center}}
\bigskip

%%%%% Begin body %%%%%%%%%%%%%%%%%%%%%%%%%%%%%%%%%%%%%%%%%%%%%%%%%%%

\begin{quote}
My goal is to establish and solve the stochastic optimal control problem that 
captures the momentum and order imbalance dynamics of the Limit Order Book 
(LOB). The solution will yield an optimal trading strategy that will permit
statistical arbitrage of the underlying stock, which will then be backtested on
historical data.
\end{quote}

\section*{Progress Timeline}
\vspace{-10pt}
\begin{table}[H]
\renewcommand\arraystretch{1.4}\arrayrulecolor{LightSteelBlue3}
\newcommand{\foo}{\color{LightSteelBlue3}\makebox[0pt]{\textbullet}\hskip-0.5pt\vrule width 1pt\hspace{\labelsep}}
\newcommand{\fooo}{\color{LightSteelBlue3}\hskip-0.5pt\vrule width 1pt\hspace{\labelsep}}
\begin{tabular}{@{\,}r <{\hskip 2pt} !{\foo} >{\raggedright\arraybackslash}p{9cm} !{\fooo} >{\raggedright\arraybackslash}p{5cm}} 
\multicolumn{1}{@{\,}r <{\hskip 9pt}}{DATE} & \multicolumn{1}{l}{THESIS} & \multicolumn{1}{l}{STA4505} \\
\hline
\st{Dec 2014} & \st{Complete CTMC calibration} \\
\st{Dec 2014} & \st{Backtest naive strategies based on CTMC} & \\
\st{Jan-May} & \st{Study stochastic controls: ECE1639, STA4505} & \\
\st{Jun 5} & \st{Establish models} & \st{Exam Study} \\
\st{Jun 12} & \st{Establish performance criteria} & \st{Exam Study} \\
\st{Jun 15} & \st{Derive DPP/DPE} & \st{EXAM} \\
\st{Jun 26} & \st{Derive continuous-time equations} & \\
\st{Jul 3} & \st{Derive discrete-time equations} & \\
\st{Jul 10} & \st{Set up MATLAB numerical integration} &  \\
\st{Jul 17} & \st{Integrate functions and plot dynamics} & \st{Integrate and analyze} \\
\st{Jul 24} & \st{Empty promises; zero fucks given} & \\
\st{Jul 31} & \st{Code for numerical solutions, calibrations, plots} & \st{Simulate results} \\
\st{Aug 7} & \st{All in-sample backtests} & \\
\st{Aug 14} & \st{All out-of-sample backtests} & \\
Aug 21 & The Big Writeup Effort & \\
{\bf Aug 24} & {\bf Submit first draft for review by panel} & \\
Aug 28 & & Project Report \\
Sep 2 & & Presentation
\end{tabular}
\end{table}

\newpage
\section*{Whiteboard Inspirational Quote of the Week}

\section*{For Our Colleagues at Duke University...}
\paragraph{Mon 10th:} holy fuck two weeks today is the unofficial submission date. 

\paragraph{Tue 11th:} this morning i met up with Aerospace Adriana. she had texted me a couple days ago to say she was applying for the Forces, which i correctly guessed was for their aerospace engineering officer position, so i wanted to make sure she knew what she was getting herself into, what to expect from the application process and initially on joining, etc. turns out she literally knows nothing about anything. she didn't even bother really getting the job description, watching the little promo videos the CF has created for each job position. fucking idiot. but in other news she submitted her dissertation a month ago and her prof still hasn't even looked at it... so i dunno, i'm not even really that far behind then... \par 
i dunno whether you bother flipping through my actual thesis pdf when i attach it, it's pretty long so i imagine you probably did the first time and now it's like ``ehhh already seen it...'' so maybe i ought to just tell you what's new in it? or just stop attaching it! \par 
tentative meeting set up with Gabe for Monday morning. last i had talked to him, i hadn't even started doing the discrete time derivations. we're talking, i dunno, early July. holy fuck, i just realized that i've actually gotten a lot done. i still need a secondary examiner though. i contacted one prof by e-mail over the weekend and he has yet to reply, so next step will be dropping into his office, taking the foot in the door analogy to the next level. 

\paragraph{Wed 12th:} yesterday i found a file on my drive that i'd clearly written after attending some inspirational `welcome to grad school, here is some advice' type lecture. i had made myself an ambitious fucking schedule that i never kept, set some goals that i never attained, and wrote down tips from that lecture. here they are:
\begin{itemize}[topsep=0pt,itemsep=0ex,partopsep=0ex,parsep=0ex]
\item work around 50hrs/week
\item have a to-do list for EVERYTHING with H M L
\item study time/location free from disruptions
\item handle things only once! reply right away or never!
\item time journal, how time gets spent
\item 10min each night to make to-do list for tomorrow
\item study: 2-3h outside class for 1h inside
\item study as soon after lecture as possible
\item max 2h for one course at a time
\end{itemize}
in particular i was struck by this idea of taking 10min each night to make a to-do list for the following day. just struck me as a really solid idea. so i gave it a shot, left myself a little gedit window open to find the next day, where i said good morning you little twat, here's what's on tap for today. and man, gotta say, i hit every single one of them, and today is definitely ranking number one in terms of crossing off items on the list below. i dig it. to boot, it's only 5:45pm and i can quite confidently call it a day! \par
oh, some other great news. one is that i've got my secondary examiner confirmed: Prof Chi-Guhn Lee from MIE. splendid. and the second is that Gabe confirmed our meeting for Monday at the Hot Oven Bakery by his place out west on Bloor and Royal York. i fucking love this bakery. shitty fucking diner coffee, and some of the BEST quiches and meat pies, at the best prices ever. i haven't delved into their baked goods yet, going to give it a shot this time. anyway, the meeting confirmation is good news for other reasons than my indulgence-to-be. I asked whether he wanted me to send over my updates beforehand, or just bring it on Mon, to which he didn't give a direct answer, suggesting he's fine with bringing it. the last thing he saw from me was a TMW from early July. there is absolutely no fucking way he's ready for the shit i'm about to drop on him, and that delights me.

\paragraph{Thu 13th:} a splendid day today. woke up, stretched a bit, had the ol smoothie, rolled back into bed.

\paragraph{Fri 14th:} i cannot fucking believe i'm looking at so many crossed-off items in the timeline on the first page. literally cannot believe. whereas i can believe it about the list on the last page. and i think this comes down to having lofty, broad items versus individual, very attainable items. today i can truly say i have learned first-hand what that means. i remember in SEALFIT training our man Cdr Mark Divine (and this is amazing that i can make this reference and we've both actually met the guy. nuts.) talks about how to get through the shitty times, like when you're fucking beat, tired hungry thirsty with no respite from those any time soon, sitting in the water, soaked, uncomfortable ... he says it's all about small victories. it's all about getting through one small little thing at a time and counting those wins. not telling yourself `i am going to make it through this weekend,' but `i am going to make it through the next 10 minutes of this task.' i feel like there's a strong analogy here. \par 
but anyway, here we are man, in pretty top shape i dare say. everything is looking fucking good to make that Aug Two Four deadline. can't wait to get to writing some top grade bullshit commentaries on the various plots that are currently just deposited in the thesis. \par 
and then can't wait to have to revise the whole goddamn thing.

\section*{The Academic Week in Review}
c.f. last week, I now unveil: `a really solid todo list that i can look at and not have any individual thing be crazy daunting.'

\clearpage 

\begin{itemize}[topsep=0pt,itemsep=0ex,partopsep=0ex,parsep=0ex]
\item \st{in-sample backtest, same-day calibration}
\begin{itemize}[topsep=0pt,itemsep=0ex,partopsep=0ex,parsep=0ex]
\item \st{re-run stoch tests to break out number of orders into MO/LO (nah fuck that)}
\item \st{reprint results table}
\item \st{plot EOD results}
\item \st{writeup}
\end{itemize}

\item \st{in-sample backtest, week offset calibration}
\begin{itemize}[topsep=0pt,itemsep=0ex,partopsep=0ex,parsep=0ex]
\item \st{finish running overnight backtest}
\item \st{print results table}
\item \st{plot EOD results}
\item \st{writeup}
\end{itemize}

\item \st{in-sample backtest, annual calibration}
\begin{itemize}[topsep=0pt,itemsep=0ex,partopsep=0ex,parsep=0ex]
\item \st{write script to produce annual calibration}
\item \st{run backtest ... should be pretty fast}
\item \st{print results table}
\item \st{plot EOD results}
\item \st{writeup}
\end{itemize}

\item out-of-sample backtest: 2014 data.
\begin{itemize}[topsep=0pt,itemsep=0ex,partopsep=0ex,parsep=0ex]
\item \st{decide WHAT to run. best performer? probably annual calib is easiest, tho. justify. also consider expanding the finite differences horizon $T$ for more accurate results...}
\item \st{run backtest. ~0.5h per strategy.}
\item print results table
\item plot EOD results
\item writeup
\end{itemize}

\item Dynamics of $\delta^\pm$ for continuous vs discrete time
\begin{itemize}[topsep=0pt,itemsep=0ex,partopsep=0ex,parsep=0ex]
\item \st{produce plots comparing for different q,z values.}
\item \st{mimic Seb's four plots to show mechanics of the FOUR methods}
\item writeup something intelligible about the plots

\end{itemize}

\item \st{Comparison of strategy performances}
\begin{itemize}[topsep=0pt,itemsep=0ex,partopsep=0ex,parsep=0ex]
\item \st{produce plot of performance for ONE DAY}
\item \st{writeup something intelligible about the plots}
\end{itemize}

\item General writing stuff
\begin{itemize}[topsep=0pt,itemsep=0ex,partopsep=0ex,parsep=0ex]
\item \st{write introduction}
\item \st{write algo trading section}
\item \st{write LOB dynamics section}
\item write ITCH data section
\item write intro to stochastic chapter
\item Link STA4505 project into dissertation with lead-in blurb.
\item re-write lead-in to continuous time section
\item re-write lead-in to discrete time section
\item write abstract
\end{itemize}

\item any remaining time to proof-read the bitch
\item {\bf SUBMIT THE BITCH}
\item write acknowledgement
\item write dedication
\item STA4505 project standalone write-up.
\end{itemize}

\end{document}