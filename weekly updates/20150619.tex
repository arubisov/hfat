% LaTeX set-up adapted from a template by Alan T. Sherman (9/9/98)
%%%%%%%%%%%%%%%%%%%%%%%%%%%%%%%%%%%%%%%%%%%%%%%%%%%%%%%%%%%%%%%%%%%%%%
\documentclass[12pt]{article}
\input{/home/anton/Documents/latex/LaTeXHeader.tex} % put hard path to header.
\usepackage{soul}
\newtheorem{theorem}{Theorem}

%%%%%% Begin document with header and title %%%%%%%%%%%%%%%%%%%%%%%%%

\begin{document}
\mymascheader
\pagestyle{plain}
{\begin{center} {\large {\bf High-Frequency Algorithmic Trading \\ with Momentum and Order Imbalance}} \end{center}}
\bigskip

%%%%% Begin body %%%%%%%%%%%%%%%%%%%%%%%%%%%%%%%%%%%%%%%%%%%%%%%%%%%

\begin{quote}
My goal is to establish and solve the stochastic optimal control problem that 
captures the momentum and order imbalance dynamics of the Limit Order Book 
(LOB). The solution will yield an optimal trading strategy that will permit
statistical arbitrage of the underlying stock, which will then be backtested on
historical data.
\end{quote}

\section*{Progress Timeline}
\begin{table}[H]
\renewcommand\arraystretch{1.4}\arrayrulecolor{LightSteelBlue3}
\newcommand{\foo}{\color{LightSteelBlue3}\makebox[0pt]{\textbullet}\hskip-0.5pt\vrule width 1pt\hspace{\labelsep}}
\newcommand{\fooo}{\color{LightSteelBlue3}\hskip-0.5pt\vrule width 1pt\hspace{\labelsep}}
\begin{tabular}{@{\,}r <{\hskip 2pt} !{\foo} >{\raggedright\arraybackslash}p{9cm} !{\fooo} >{\raggedright\arraybackslash}p{5cm}} 
\multicolumn{1}{@{\,}r <{\hskip 9pt}}{DATE} & \multicolumn{1}{l}{THESIS} & \multicolumn{1}{l}{STA4505} \\
\hline
\st{Dec 2014} & \st{Complete CTMC calibration} \\
\st{Dec 2014} & \st{Backtest naive strategies based on CTMC} & \\
\st{Jan-May} & \st{Study stochastic controls: ECE1639, STA4505} & \\
\st{Jun 5} & \st{Establish models} & \st{Exam Study} \\
\st{Jun 12} & \st{Establish performance criteria} & \st{Exam Study} \\
\st{Jun 15} & Derive DPP/DPE & \st{EXAM} \\
Jun 19 & Derive DPP/DPE & \\
Jun 26 & Derive DPE/Solve PDEs & \\
Jul 3 & Solve PDEs & \\
Jul 10 & Solve PDEs and implement numerical solution & Numerical solution \\
Jul 31 & Backtest solution on historical data & Implement simulations \\
Aug 15 & Dissertation Writeup & Project Writeup \\
\end{tabular}
\end{table}

\newpage

\section*{For Our Readers in the Middle East...}
 

\section*{The Academic Week in Review}

\subsection*{The Dissertation}
Okay at this point I'd like to cycle back and re-jig the problem setup that we've done thus far. Largely this is due to Tue morning's meeting with Sebastian at the cafe: whereas I was convinced I was dealing with a \textbf{delayed} stochastic differential equation, which would require sifting through another textbook to understand, turns out this is far from the truth. To recap: previously I was concerned that I needed $Z_t = (\rho_t, \Delta_t) $ and $\zeta_t = Z_{t-s} = (\rho_{t-s}, \Delta_{t-s}) $ terms in my value function, because in the exploratory data analysis we had derived some conditional probabilities that informed us that \textbf{if} we had seen a price change, and the imbalance now is the same as the imbalance before, \textbf{then} we would expect another equal price change. But! These conditional probabilities are actually \textit{embedded within the CTMC generator $G$}, which is for the couplet at time $t$, it doesn't require a time-delay. Hence, we don't need this lookback/delay $\zeta_t$ term at all, and so it's smooth sailing to proceed.

Below we list the processes involved in the optimization problem:

\begin{tabular}{lll}
Imbalance \& Midprice Change & $\bZ_t = (\rho_t, \Delta_t) $ & CTMC with generator $G$ \\
Imbalance & $\rho_t = \bZ_t^{(1)}$ & LOB imbalance at time $t$ \\
Midprice & $S_t$ & evolves according to CTMC \\
Midprice Change & $\Delta_t = \bZ_t^{(2)} = S_t - S_{t-s}$ & $s$ a pre-determined interval \\
Bid-Ask half-spread & $\pi_t$ & constant? \\
LOB Shuffling & $N_t$ & Poisson with rate $\lambda(\bZ_t)$ \\
$\Delta\text{Price:}$ LOB shuffled & $\{ \eta_{0,z}, \eta_{1,z}, \dots \} \sim F_{z}$ & i.i.d. with $z = (k,l)$, where \\
& & $k \in \{ \text{\#bins} \}, \; l \in \{ \Delta \$ \}$ \\
Other Agent MOs & $K^{\pm}_t$ & Poisson with rate $\mu^{\pm}(\bZ_t)$ \\
LO posted depth & $\delta^{\pm}_t$ & our $\cF$-predictable controlled processes \\
Our LO fill count & $L^{\pm}_t$ & $\cF$-predictable, non-Poisson \\
Our MOs & $M^{\pm}_t$ & our controlled counting process \\
Our MO execution times & $\btau = \{ \tau_k : k = 1, \dots, K \}$ & increasing sequence of $\cF$-stopping times \\
Cash & $X^{\btau, \delta}_t$ & depends on our processes $M$ and $\delta$ \\
Inventory & $Q^{\btau, \delta}_t$ & depends on our processes $M$ and $\delta$ \\
\end{tabular}

$L^{\pm}_t$ are counting processes (not Poisson) satisfying the relationship that if at time $t$ we have a sell limit order posted at a depth $\delta^{-}_t$, then our fill probability is $e^{-\kappa \delta^{-}_t}$ conditional on a buy market order arriving; namely:
\[ \P [\d L^{-}_t = 1 \, | \, \d K^+_t = 1] = e^{-\kappa \delta^{-}_t} \]
\[ \P [\d L^{+}_t = 1 \, | \, \d K^{-}_t = 1] = e^{-\kappa \delta^{+}_t} \]

The midprice $S_t$ evolves according to the Markov chain and hence is Poisson with rate $\lambda$ and jump size $\eta$, both of which depend on the state of the Markov chain. This Poisson process is all-inclusive in the sense that it accounts for any midprice change, be it from executions, cancellations, or order modifications with the LOB. Thus, the stock midprice $S_t$ evolves according to the SDE:
\begin{equation}\label{eq:stockprocess}
\d S_t = \eta_{N_{t^-},Z_{t^-}} \d N_t
\end{equation}
and additionally satisfies:
\begin{equation}\label{eq:stockintegral} 
S_t = S_{t_0} + \int\limits_{t_0}^t \Delta_u \du
\end{equation}

In executing market orders, we assume that the size of the MOs is small enough to achieve the best bid/ask price, and not walk the book. Hence, our cash process evolves according to:
\begin{equation}\label{eq:cashprocess}
\begin{split}
\d X^{\btau, \delta}_t = 	&\underbrace{(S_t + \pi_t + \delta^{-}_t) \d L^{-}_t}_{\text{sell limit order}} - \underbrace{(S_t - \pi_t - \delta^{+}_t) \d L^{+}_t}_{\text{buy limit order}} \\
						&+ \underbrace{(S_t - \pi_t) \d M^{-}_t}_{\text{sell market order}} - \underbrace{(S_t + \pi_t) \d M^{+}_t}_{\text{buy market order}}
\end{split}
\end{equation}

Based on our execution of limit and market orders, our inventory satisfies:
\begin{equation}\label{eq:inventory}
Q^{\btau, \delta}_0 = 0, \qquad Q^{\btau, \delta}_t = L^+_t + M^+_t - L^-_t - M^-_t
\end{equation}

We define a new variable for our net present value (NPV) at time $t$, call it $W^{\btau, \delta}_t$, and hence $W^{\btau, \delta}_T$ at terminal time $T$ is our `terminal wealth'. In algorithmic trading, we want to finish the trading day with zero inventory, and assume that at the terminal time $T$ we will submit a market order (of a possibly large volume) to liquidate remaining stock. Here we do not assume that we can receive the best bid/ask price - instead, the price achieved will be $(S - \mathrm{sign}(Q)\pi - \alpha Q)$, where $\mathrm{sign}(Q)\pi$ represents crossing the spread in the direction of trading, and $\alpha Q$ represents receiving a worse price linearly in $Q$ due to walking the book. Hence, $W^{\btau, \delta}_t$ satisfies:
\begin{equation}
\label{eq:terminalwealth}
W^{\btau, \delta}_t = \underbrace{\vphantom{\left( Q^{\btau, \delta}_t) \right)}X^{\btau, \delta}_t}_{\text{cash}}+ \underbrace{Q^{\btau, \delta}_t \left( S_t - \mathrm{sign}(Q^{\btau, \delta}_t)\pi_t \right)}_{\text{book value of assets}} - \underbrace{\alpha \left( Q^{\btau, \delta}_t \right)^2}_{\text{liquidation penalty}}
\end{equation}

The set of admissible trading strategies $\cA$ is the set of all $\cF$-stopping times and $\cF$-predictable, bounded-from-below depths $\delta$. 

For deriving an optimal trading strategy, I will consider three separate performance criteria, which allow us to evaluate the performance of a given strategy:
\begin{enumerate}[noitemsep, topsep=0pt]
\item Profit: $H^{\btau, \delta}(\cdot) = \E \left[ W_T^{\btau, \delta} \right]\vphantom{\int\limits_t^T}$
\item Profit with risk aversion: $H^{\btau, \delta}(\cdot) = \E \left[  W_T^{\btau, \delta} - \gamma \mathbf{1}_{W_T^{\btau, \delta}<0} \cdot W_T^{\btau, \delta} \right]\vphantom{\int\limits_t^T}$
\item Profit with running inventory penalty: $H^{\btau, \delta}(\cdot) = \E \left[  W_T^{\btau, \delta}  - \varphi \int\limits_t^T \left( Q^{\btau, \delta}_u \right)^2 \du  \right]$
\end{enumerate}

In each of the cases, the value function is given by
\begin{equation}\label{eq:valuefunction}
H(t,x,s,\bz,q) = \sup\limits_{\btau \in \mathcal{T}_{[t,T]}} \sup\limits_{\delta \in \cA_{[t,T]}} H^{\btau, \delta}(t,x,s,\bz,q)
\end{equation}

\subsection*{Dynamic Programming Principle for Optimal Stopping and Control}
\begin{theorem}
If an agent's performance criteria for a given admissible control $\bu$ and admissible stopping time $\tau$ are given by
\[ H^{\tau, \bu}(t,\bx) = \E_{t,\bx} [ G (X^{\bu}_\tau)] \]
and the value function is
\[ H(t,\bx) = \sup\limits_{\tau \in \mathcal{T}_{[t,T]}} \sup\limits_{\bu \in \cA_{[t,T]}} H^{\tau, \bu}(t,\bx) \]
then the value function satisfies the Dynamic Programming Principle
\begin{equation}
\label{eq:thmDPP}
H(t,\bx) = \sup\limits_{\tau \in \mathcal{T}_{[t,T]}} \sup\limits_{\bu \in \cA_{[t,T]}} \E_{t,\bx} \left[ G (X^{\bu}_\tau) \indicator_{\tau<\theta} + H(\theta, X^{\bu}_\theta)\indicator_{\tau \geq \theta} \right]
\end{equation}
for all $(t,\bx) \in [0,T] \times \R^m$ and all stopping times $\theta \leq T$.
\end{theorem}

\subsection*{Dynamic Programming Equation for Optimal Stopping and Control}
\begin{theorem}
Assume that the value function $H(t,\bx)$ is once differentiable in $t$, all second-order derivatives in $\bx$ exist, and that \myfunction{G}{\R^m}{\R} is continuous. Then $H$ solves the quasi-variational inequality
\begin{equation}
\label{eq:thmDPE}
\max \left\lbrace \partial_t H + \sup \limits_{\bu \in \cA_t} \cL^{\bu}_t H \; ; \; G - H \right\rbrace = 0
\end{equation}
on $\mathcal{D}$, where $\mathcal{D} = [0,T] \times \R^m$.
\end{theorem}

\subsection*{Maximizing Profit}
Ok let's get to business. We need to solve the DPE that results from using the first performance criteria in our value function. So our $G$ function is exactly our NPV term $W$, and really the work comes from finding the infinitesimal generator for the processes. Let's get on it.

The quasi-variational inequality in equation \ref{eq:thmDPE} can be interpreted as follows: the max operator is choosing between posting limit orders or executing market orders; the second term, $G-H$, is the stopping region and represents the value derived from executing a market order; and the first term is the continuation region, representing the value of posting limit orders. We'll use the shorthand $H(\cdot) = H(t,x,s,\bz,q)$ and solve for $\d H$ inside the continuation region, hence $\d M^{\pm} = 0$, in order to then extract out the infinitesimal generator.

\begin{align*}
\d H (t,x,s,\bz,q) & = \sum\limits_i \partial_{x_i} H \dx_i \\
& = \partial_t H \dt + \partial_{K^{\pm}} H \d {K^{\pm}} + \partial_{\bZ} H \d {\bZ} \\
& = \partial_t H \dt + \bigl\lbrace e^{ -\kappa \delta^{-}} \E \bigl[ H(t,x+(s+\pi+\delta^-),s,\bz,q-1) - H(\cdot) \bigr] \bigr\rbrace \d K^+\\
& \hphantom{{}={} \partial_t H \dt} + \bigl\lbrace e^{ -\kappa \delta^{+}} \E \bigl[ H(t,x-(s-\pi-\delta^+),s,\bz,q+1) - H(\cdot) \bigr] \bigr\rbrace \d K^-\\
& \hphantom{{}={}} + \sum_{k\in P} \sum_{l \in \{-1,0,1\}} \E \left[ H(t,x,s+\mathrm{sign}(l)\eta_{0,\bz},\bz+(k,l),q) - H(\cdot) \right] \d Z_{\bz,(k,l)}
\end{align*}
Substitute in the following identities for the compensated processes
\begin{align*} 
\d M^{\pm} & = \d \tilde{K}^{\pm} + \mu^{\pm}(\bz) \dt \\
\d Z_{\bz,(k,l)}  & = \d \tilde{Z}_{\bz,(k,l)}  + G_{\bz,(k,l)} \dt 
\end{align*}
\begin{align*}
{}\phantom{\d H (t,x,s,\bz,q)} & = \partial_t H \dt + \biggl\lbrace \mu^+(\bz) e^{ -\kappa \delta^{-}} \E \bigl[ H(t,x+(s+\pi+\delta^-),s,\bz,q-1) - H(\cdot) \bigr] \\
& \hphantom{{}={} \partial_t H \dt +} + \mu^-(\bz) e^{ -\kappa \delta^{+}} \E \bigl[ H(t,x-(s-\pi-\delta^+),s,\bz,q+1) - H(\cdot) \bigr] \\
& \hphantom{{}={}} + \sum_{k\in P} \sum_{l \in \{-1,0,1\}} G_{\bz,(k,l)} \E \left[ H(t,x,s+\mathrm{sign}(l)\eta_{0,\bz},\bz+(k,l),q) - H(\cdot) \right]  \biggr\rbrace \dt \\
& \hphantom{{}={} \partial_t H \dt} + \bigl\lbrace e^{ -\kappa \delta^{-}} \E \bigl[ H(t,x+(s+\pi+\delta^-),s,\bz,q-1) - H(\cdot) \bigr] \bigr\rbrace \d \tilde{K}^+\\
& \hphantom{{}={} \partial_t H \dt} + \bigl\lbrace e^{ -\kappa \delta^{+}} \E \bigl[ H(t,x-(s-\pi-\delta^+),s,\bz,q+1) - H(\cdot) \bigr] \bigr\rbrace \d \tilde{K}^-\\
& \hphantom{{}={}} + \sum_{k\in P} \sum_{l \in \{-1,0,1\}} \E \left[ H(t,x,s+\mathrm{sign}(l)\eta_{0,\bz},\bz+(k,l),q) - H(\cdot) \right] \d \tilde{Z}_{\bz,(k,l)}\
\end{align*}
From which we can see that the infinitesimal generator is given by
\begin{equation}
\begin{split}
\cL^{\delta}_t H & = \mu^+(\bz) e^{ -\kappa \delta^{-}} \E \bigl[ H(t,x+(s+\pi+\delta^-),s,\bz,q-1) - H(\cdot) \bigr] \\
& \quad + \mu^-(\bz) e^{ -\kappa \delta^{+}} \E \bigl[ H(t,x-(s-\pi-\delta^+),s,\bz,q+1) - H(\cdot) \bigr] \\
& \quad + \sum_{k\in P} \sum_{l \in \{-1,0,1\}} G_{\bz,(k,l)} \E \left[ H(t,x,s+\mathrm{sign}(l)\eta_{0,\bz},\bz+(k,l),q) - H(\cdot) \right] 
\end{split}
\end{equation}


\section*{Looking Ahead}

\end{document}
