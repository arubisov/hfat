% LaTeX set-up adapted from a template by Alan T. Sherman (9/9/98)
%%%%%%%%%%%%%%%%%%%%%%%%%%%%%%%%%%%%%%%%%%%%%%%%%%%%%%%%%%%%%%%%%%%%%%
\documentclass[12pt]{article}
\input{/home/anton/Documents/latex/LaTeXHeader.tex} % put hard path to header.
\usepackage{soul}
\newtheorem{theorem}{Theorem}
\usepackage{hyperref}

%%%%%% Begin document with header and title %%%%%%%%%%%%%%%%%%%%%%%%%

\begin{document}
\mymascheader
\pagestyle{plain}
{\begin{center} {\large {\bf High-Frequency Algorithmic Trading \\ with Momentum and Order Imbalance}} \end{center}}
\bigskip

%%%%% Begin body %%%%%%%%%%%%%%%%%%%%%%%%%%%%%%%%%%%%%%%%%%%%%%%%%%%

\begin{quote}
My goal is to establish and solve the stochastic optimal control problem that 
captures the momentum and order imbalance dynamics of the Limit Order Book 
(LOB). The solution will yield an optimal trading strategy that will permit
statistical arbitrage of the underlying stock, which will then be backtested on
historical data.
\end{quote}

\section*{Progress Timeline}
\begin{table}[H]
\renewcommand\arraystretch{1.4}\arrayrulecolor{LightSteelBlue3}
\newcommand{\foo}{\color{LightSteelBlue3}\makebox[0pt]{\textbullet}\hskip-0.5pt\vrule width 1pt\hspace{\labelsep}}
\newcommand{\fooo}{\color{LightSteelBlue3}\hskip-0.5pt\vrule width 1pt\hspace{\labelsep}}
\begin{tabular}{@{\,}r <{\hskip 2pt} !{\foo} >{\raggedright\arraybackslash}p{9cm} !{\fooo} >{\raggedright\arraybackslash}p{5cm}} 
\multicolumn{1}{@{\,}r <{\hskip 9pt}}{DATE} & \multicolumn{1}{l}{THESIS} & \multicolumn{1}{l}{STA4505} \\
\hline
\st{Dec 2014} & \st{Complete CTMC calibration} \\
\st{Dec 2014} & \st{Backtest naive strategies based on CTMC} & \\
\st{Jan-May} & \st{Study stochastic controls: ECE1639, STA4505} & \\
\st{Jun 5} & \st{Establish models} & \st{Exam Study} \\
\st{Jun 12} & \st{Establish performance criteria} & \st{Exam Study} \\
\st{Jun 15} & \st{Derive DPP/DPE} & \st{EXAM} \\
\st{Jun 19} & \st{Derive DPP/DPE} &  \\
\st{Jun 26} & \st{Derive continuous-time equations} & \\
\st{Jul 3} & \textcolor{red}{Derive discrete-time equations} & \\
Jul 10 & Set up MATLAB numerical integration &  \\
Jul 17 & Integrate functions and plot dynamics & Integrate and analyze too! \\
Jul 24 & More dynamics, and calib/choose parameters & \\
Jul 31 & Backtest on historical data & Simulate results \\
Aug 7 & More backtesting, comparing with previous & \\
Aug 14 & Dissertation writeup / buffer & Project writeup \\
Aug 21 & Dissertation writeup / buffer &  \\
Aug 28 & Dissertation writeup & Presentation \\
\end{tabular}
\end{table}

\newpage
\section*{Whiteboard Inspirational Quote of the Week}
\begin{quote}
\textit{} \\- 
\end{quote} 

\section*{For Our Readers in the Middle East...}

\section*{The Academic Week in Review}
Recall we are trying to solve the first case of the DPE, namely:
\begin{equation}
\label{eq:discretesup1}
\begin{split}
& \sup_{\delta^\pm} \biggl\lbrace \E_{\bw} \biggl[
(s + \pi + \delta^-)L_k^- - (s - \pi - \delta^+)L_k^+ \\
& \hphantom{\sup_{\bu} \biggl\lbrace \E_{\bw} \biggl[ {}+{}} + (L_k^+ - L_k^-) \bigl( s + \eta_{0,\bz} T(\bz, \omega)^{(2)}  -\sgn( q + L_k^+ - L_k^-)\pi   \bigr) \\
& \hphantom{\sup_{\bu} \biggl\lbrace \E_{\bw} \biggl[ {}+{}} + q\left( \eta_{0,\bz} T(\bz, \omega)^{(2)}  -\left( \sgn( q + L_k^+ - L_k^-) - \sgn(q) \right) \pi \right) \\
& \hphantom{\sup_{\bu} \biggl\lbrace \E_{\bw} \biggl[ {}+{}} + h_{k+1}(T(\bz,\omega), q + L_k^+ - L_k^- ) -  h_k(\bz,q) \biggr] \biggr\rbrace
\end{split}
\end{equation}
You might recall that things got pretty messy pretty fast. Previously we had set up the problem such that at each timestep $k$ there can be multiple other agents' market orders ($K_k^+$) arriving, and these were Poisson distributed. For each arriving order, the probability of our posted limit order being filled was $e^{-\kappa \delta^-}$. We're going to modify this slightly. Keeping the market orders as Poisson distributed, we have that $\P [ K_k^+ = 0] = \frac{e^{-\lambda \Delta t} (\lambda \Delta t)^0}{0!} = e^{-\lambda \Delta t}$, and so the probability of seeing some positive number of market orders is
\begin{equation}
\label{eq:discretepositiveKplus}
\P [ K_k^+ > 0] = 1 - e^{-\lambda \Delta t}
\end{equation}
Now we make the simplified assumption that the \textit{aggregate} of the orders walks the limit order book to a depth of $p_k$, and if $p_k > \delta^-$, then our sell limit order is lifted. Thus we have the following preliminary results:
\begin{align*}
\P [ L_k^- = 1 | K_k^+ > 0] & = e^{-\kappa \delta^-} \\
\P [ L_k^- = 0 | K_k^+ > 0] & = 1 - e^{-\kappa \delta^-} \\
\E [ L_k^- ] & = \P [ L_k^- = 1 | K_k^+ > 0] \cdot \P[K_k^+ > 0] \\
& = ( 1 - e^{-\lambda \Delta t} ) e^{-\kappa \delta^-} \\
\end{align*}
For ease of notation, we'll write the probability of the $L_k^- = 1$ event as $p(\delta^-)$. This gives us the additional results:
\begin{align*}
\P [ L_k^- = 1] & = p(\delta^-) = \E [ L_k^-] \\
\P [ L_k^- = 0] & = 1 - p(\delta^-) \\
\partial_{\delta^-} \P [ L_k^- = 1]  & = -\kappa p(\delta^-) \\
\partial_{\delta^-} \P [ L_k^- = 0] & = \kappa p(\delta^-) \\
\end{align*}
Let's pre-compute some of the terms that we'll encounter in the supremum, namely the expectations of the random variables. 
\begin{align}
\E [\sgn(q + L_k^+ - L_k^-)] & = \P[L_k^- = 1] \cdot \P[L_k^+ = 1] \cdot \sgn(q) \nonumber \\
& \quad + \P[L_k^- = 1] \cdot \P[L_k^+ = 0] \cdot \sgn(q - 1) \nonumber \\
& \quad +  \P[L_k^- = 0] \cdot \P[L_k^+ = 1] \cdot \sgn(q+1) \nonumber \\
& \quad + \P[L_k^- = 0] \cdot \P[L_k^+ = 0] \cdot \sgn(q) \nonumber \\
& = p(\delta^-)p(\delta^+) \sgn(q) \nonumber \\
& \quad + p(\delta^-) (1-p(\delta^+)) \sgn(q - 1) \nonumber \\
& \quad + (1 - p(\delta^-)) p(\delta^+)  \sgn(q+1) \nonumber \\
& \quad + (1 - p(\delta^-)) (1-p(\delta^+))  \sgn(q) \nonumber \\
& = \sgn(q) \bigl[ 1 - p(\delta^+) - p(\delta^-) + 2 p(\delta^+) p(\delta^-) \bigr] \nonumber \\
& \quad + \sgn(q-1) \bigl[ p(\delta^-)  - p(\delta^+) p(\delta^-) \bigr]  \nonumber \\
& \quad + \sgn(q+1) \bigl[ p(\delta^+)  - p(\delta^+) p(\delta^-) \bigr]  \nonumber \\
& = \begin{cases} 
1 & q \geq 2 \\
1 - p(\delta^-)(1 - p(\delta^+)) & q = 1 \\
p(\delta^+) - p(\delta^-) & q = 0 \\
-\bigl[ 1 - p(\delta^+)(1 - p(\delta^-)) \bigr] & q = -1 \\
-1 & q \leq -2
\end{cases} \\
& = \Phi(q, \delta^+, \delta^-)
\end{align}
Similarly:
\begin{align}
\E [ L_k^+ \sgn(q + L_k^+ - L_k^-)] & = \P[L_k^- = 1] \cdot \P[L_k^+ = 1] \cdot \sgn(q) \nonumber \\
& \quad + \P[L_k^- = 1] \cdot \P[L_k^+ = 0] \cdot 0 \sgn(q - 1) \nonumber \\
& \quad +  \P[L_k^- = 0] \cdot \P[L_k^+ = 1] \cdot \sgn(q+1) \nonumber \\
& \quad + \P[L_k^- = 0] \cdot \P[L_k^+ = 0] \cdot 0 \sgn(q) \nonumber \\
& = p(\delta^+) \bigl[ p(\delta^-) \sgn(q) + (1-p(\delta^-) \sgn(q+1) \bigr] \nonumber \\
& = p(\delta^+) \begin{cases} 
1 & q \geq 2 \\
1 & q = 1 \\
(1 - p(\delta^-)) & q = 0 \\
-p(\delta^-) & q = -1 \\
-1 & q \leq -2
\end{cases} \\
& = p(\delta^+) \Psi(q, \delta^-)
\end{align}
and
\begin{align}
\E [ L_k^- \sgn(q + L_k^+ - L_k^-)] & = p(\delta^-) \bigl[ p(\delta^+) \sgn(q) + (1-p(\delta^+)) \sgn(q-1) \bigr] \nonumber \\
& = p(\delta^-) \begin{cases} 
1 & q \geq 2 \\
p(\delta^+) & q = 1 \\
-(1 - p(\delta^+)) & q = 0 \\
-1 & q = -1 \\
-1 & q \leq -2
\end{cases} \\
& = p(\delta^-) \Upsilon(q, \delta^+)
\end{align}
We'll also require the partial derivatives of these expectations, which we can easily compute. Below we'll use the simplified notation $\Phi_+$ to denote the function closely associated with the partial derivative of $ \Phi$ with respect to $\delta^+$.
\begin{align}
\partial_{\delta^-} \E [\sgn(q + L_k^+ - L_k^-)] = \partial_{\delta^-} \Phi(q, \delta^+, \delta^-) & = \kappa p(\delta^-) \begin{cases} 
0 & q \geq 2 \\
(1-p(\delta^+)) & q = 1 \\
1 & q = 0 \\
p(\delta^+)  & q = -1 \\
0 & q \leq -2
\end{cases} \\
& = \kappa p(\delta^-) \Phi_-(q,\delta^+) \\
\partial_{\delta^+} \E [\sgn(q + L_k^+ - L_k^-)] =\partial_{\delta^+} \Phi(q, \delta^+, \delta^-) & =  \kappa p(\delta^+) \begin{cases} 
0 & q \geq 2 \\
- p(\delta^-) & q = 1 \\
- 1 & q = 0 \\
- (1 - p(\delta^-)) & q = -1 \\
0 & q \leq -2
\end{cases} \\
& = \kappa p(\delta^+)\Phi_+(q,\delta^-) \\
\partial_{\delta^-} \E [L_k^+ \sgn(q + L_k^+ - L_k^-)] = \partial_{\delta^-} p(\delta^+) \Psi(q, \delta^-) & = \kappa p(\delta^+) p(\delta^-)\begin{cases} 
0 & q \geq 2 \\
0 & q = 1 \\
1 & q = 0 \\
1 & q = -1 \\
0 & q \leq -2
\end{cases} \\
& = \kappa p(\delta^+) p(\delta^-) \Psi_-(q) \\
\partial_{\delta^+} \E [L_k^+ \sgn(q + L_k^+ - L_k^-)] = \partial_{\delta^+} p(\delta^+) \Psi(q, \delta^-) & = -\kappa p(\delta^+) \Psi(q, \delta^-)
 \\
\partial_{\delta^-} \E [L_k^- \sgn(q + L_k^+ - L_k^-)] = \partial_{\delta^-} p(\delta^-) \Upsilon(q, \delta^+) & = -\kappa p(\delta^-) \Upsilon(q, \delta^+)
 \\
 \partial_{\delta^+} \E [L_k^- \sgn(q + L_k^+ - L_k^-)] = \partial_{\delta^+} p(\delta^-) \Upsilon(q, \delta^+)& = \kappa p(\delta^+) p(\delta^-) \begin{cases} 
0 & q \geq 2 \\
-1 & q = 1 \\
-1 & q = 0 \\
0 & q = -1 \\
0 & q \leq -2
\end{cases} \\
& = \kappa p(\delta^+) p(\delta^-)\Upsilon_+(q)
\end{align}
Recalling that we have  $\bP$ the transition matrix for the Markov Chain $\bZ$, with $\bP_{\bz, \bj} = \P[\bZ_{k+1} = \bj | \bZ_k = \bz]$, then we can also write:
\begin{equation}
\begin{split}
\E[h_{k+1}(T(\bz,\omega), q + L_k^+ - L_k^- )] & = \sum_\bj \bP_{\bz,\bj} \biggl[ h_{k+1}(\bj, q) \bigl[ 1 - p(\delta^+) - p(\delta^-) + 2 p(\delta^+) p(\delta^-) \bigr]  \\
& \hphantom{\sum_\bj \bP_{\bz,\bj} \biggl[ {}+{}} + h_{k+1}(\bj, q-1) \bigl[ p(\delta^-)  - p(\delta^+) p(\delta^-) \bigr]   \\
& \hphantom{\sum_\bj \bP_{\bz,\bj} \biggl[ {}+{}} + h_{k+1}(\bj, q+1) \bigl[ p(\delta^+)  - p(\delta^+) p(\delta^-) \bigr] \biggr]  \\
\end{split}
\end{equation}
and its partial derivatives as
\begin{align}
\begin{split}
\partial_{\delta^-} \E[h_{k+1}(T(\bz,\omega), q + L_k^+ - L_k^- )] & = 
\sum_\bj \bP_{\bz,\bj} \biggl[ h_{k+1}(\bj, q) \bigl[ \kappa p(\delta^-) - 2 \kappa p(\delta^+) p(\delta^-) \bigr]  \\
& \hphantom{\sum_\bj \bP_{\bz,\bj} \biggl[ {}+{}} + h_{k+1}(\bj, q-1) \bigl[ -\kappa p(\delta^-) + \kappa p(\delta^+) p(\delta^-) \bigr]   \\
& \hphantom{\sum_\bj \bP_{\bz,\bj} \biggl[ {}+{}} + h_{k+1}(\bj, q+1) \bigl[ \kappa p(\delta^+) p(\delta^-) \bigr] \biggr]
\end{split} \\
\begin{split}
& = \kappa p(\delta^-) \sum_\bj \bP_{\bz,\bj} \biggl[ h_{k+1}(\bj, q) \bigl[ 1 - 2 p(\delta^+) \bigr]  \\
& \hphantom{d(\delta^-) \sum_\bj \bP_{\bz,\bj} \biggl[ {}+{}} + h_{k+1}(\bj, q-1) \bigl[-1 + p(\delta^+) \bigr]   \\
& \hphantom{d(\delta^-) \sum_\bj \bP_{\bz,\bj} \biggl[ {}+{}} + h_{k+1}(\bj, q+1) \bigl[ p(\delta^+) \bigr] \biggr]
\end{split} \\
\begin{split}
\partial_{\delta^+} \E[h_{k+1}(T(\bz,\omega), q + L_k^+ - L_k^- )] & = 
\kappa p(\delta^+) \sum_\bj \bP_{\bz,\bj} \biggl[ h_{k+1}(\bj, q) \bigl[ 1 - 2 p(\delta^-) \bigr]  \\
& \hphantom{d(\delta^-) \sum_\bj \bP_{\bz,\bj} \biggl[ {}+{}} + h_{k+1}(\bj, q-1) \bigl[ p(\delta^-) \bigr]   \\
& \hphantom{d(\delta^-) \sum_\bj \bP_{\bz,\bj} \biggl[ {}+{}} + h_{k+1}(\bj, q+1) \bigl[ -1 + p(\delta^-) \bigr] \biggr]
\end{split}
\end{align}

Now we tackle solving the supremum in equation $\ref{eq:discretesup1}$. First we consider the first-order condition on $\delta^-$, namely that the partial derivative with respect to it must be equal to zero.
\begin{align}
\begin{split}
0 & = \partial_{\delta^-} \biggl\lbrace 
(s + \pi + \delta^-)\E [L_k^-] - (s - \pi - \delta^+)\E[L_k^+] \\
& \hphantom{\partial_{\delta^-} \biggl\lbrace {}+{}} + \E[L_k^+] \left( s + \E[\eta_{0,\bz} T(\bz, \omega)^{(2)}] \right)  - \pi \E \left[L_k^+ \sgn( q + L_k^+ - L_k^-) \right] \\
& \hphantom{\partial_{\delta^-} \biggl\lbrace {}+{}} - \E[L_k^-] \left( s + \E[\eta_{0,\bz} T(\bz, \omega)^{(2)}] \right) + \pi \E \left[ L_k^- \sgn( q + L_k^+ - L_k^-) \right] \\
& \hphantom{\partial_{\delta^-} \biggl\lbrace {}+{}} + q \E[ \eta_{0,\bz} T(\bz, \omega)^{(2)}]  - q \pi \E[ \sgn( q + L_k^+ - L_k^-)] + q \pi \sgn(q)  \\
& \hphantom{\partial_{\delta^-} \biggl\lbrace {}+{}} + \E \left[ h_{k+1}(T(\bz,\omega), q + L_k^+ - L_k^- ) \right] -  h_k(\bz,q) \biggr\rbrace
\end{split} \\
\begin{split}
& = \partial_{\delta^-} \biggl\lbrace 
(s + \pi + \delta^-)\E [L_k^-] - \pi \E \left[L_k^+ \sgn( q + L_k^+ - L_k^-) \right] \\
& \hphantom{\partial_{\delta^-} \biggl\lbrace {}+{}} - \E[L_k^-] \left( s + \E[\eta_{0,\bz} T(\bz, \omega)^{(2)}] \right) + \pi \E \left[ L_k^- \sgn( q + L_k^+ - L_k^-) \right] \\
& \hphantom{\partial_{\delta^-} \biggl\lbrace {}+{}} - q \pi \E[ \sgn( q + L_k^+ - L_k^-)]  + \E \left[ h_{k+1}(T(\bz,\omega), q + L_k^+ - L_k^- ) \right]  \biggr\rbrace
\end{split} \\
\begin{split}
& = p(\delta^-) - \kappa p(\delta^-) (s + \pi + \delta^-) - \pi \kappa p(\delta^+)p(\delta^-)\Psi_-(q) \\
& \hphantom{{}={}} + \kappa p(\delta^-) \left( s + \E[\eta_{0,\bz} T(\bz, \omega)^{(2)}] \right) - \pi \kappa p(\delta^-) \Upsilon(q,\delta^+) - q \pi \kappa p(\delta^-) \Phi_-(q,\delta^+) \\
& \hphantom{{}={}} + \kappa p(\delta^-) \sum_\bj \bP_{\bz,\bj} \biggl[ h_{k+1}(\bj, q) \bigl[ 1 - 2 p(\delta^+) \bigr] + h_{k+1}(\bj, q-1) \bigl[-1 + p(\delta^+) \bigr] \\
& \hphantom{{}={} + \kappa p(\delta^-) \sum_\bj \bP_{\bz,\bj} \biggl[} + h_{k+1}(\bj, q+1) \bigl[ p(\delta^+) \bigr] \biggr]
\end{split}
\end{align}
Dividing through by $\kappa p(\delta^-)$, which is nonzero, and re-arranging, we find that the optimal sell posting depth is given by
\begin{align}
\begin{split}
{\delta^-}^* & = \frac{1}{\kappa} + \E[\eta_{0,\bz} T(\bz, \omega)^{(2)}] - \pi \left( 1 + p(\delta^+)\Psi_-(q) + \Upsilon(q,\delta^+) + q \Phi_-(q,\delta^+) \right) \\
& \hphantom{{}={}} +  \sum_\bj \bP_{\bz,\bj} \biggl[ h_{k+1}(\bj, q) \bigl[ 1 - 2 p(\delta^+) \bigr] + h_{k+1}(\bj, q-1) \bigl[-1 + p(\delta^+) \bigr] + h_{k+1}(\bj, q+1) \bigl[ p(\delta^+) \bigr] \biggr]
\end{split} \\
\begin{split}
& = \frac{1}{\kappa} + \E[\eta_{0,\bz} T(\bz, \omega)^{(2)}] - 2 \pi \left( \indicator_{q \geq 1} + p(\delta^+)\indicator_{q = 0} \right) \\
& \hphantom{{}={}} +  \sum_\bj \bP_{\bz,\bj} \biggl[ h_{k+1}(\bj, q) \bigl[ 1 - 2 p(\delta^+) \bigr] + h_{k+1}(\bj, q-1) \bigl[-1 + p(\delta^+) \bigr] + h_{k+1}(\bj, q+1) \bigl[ p(\delta^+) \bigr] \biggr]
\end{split} \\
\begin{split}
& = \frac{1}{\kappa} + \E[\eta_{0,\bz} T(\bz, \omega)^{(2)}] - 2 \pi \indicator_{q \geq 1} + \sum_\bj \bP_{\bz,\bj} \biggl[ h_{k+1}(\bj, q) - h_{k+1}(\bj, q-1) \biggr] \\
& \hphantom{{}={}} -p(\delta^+) \left( 2 \pi \indicator_{q = 0} - \sum_\bj \bP_{\bz,\bj} \biggl[ h_{k+1}(\bj, q-1) + h_{k+1}(\bj, q+1) -2 h_{k+1}(\bj, q)  \biggr] \right)
\end{split}
\end{align}
\section*{Looking Ahead}
\end{document}
