% LaTeX set-up adapted from a template by Alan T. Sherman (9/9/98)
%%%%%%%%%%%%%%%%%%%%%%%%%%%%%%%%%%%%%%%%%%%%%%%%%%%%%%%%%%%%%%%%%%%%%%
\documentclass[12pt]{article}
\usepackage{amsmath}
\usepackage{amsfonts}
\usepackage{tabularx}
\usepackage{float}
\usepackage{color}

% SPECIFIC TO THE TIMELINE
\usepackage[utf8]{inputenc}
\usepackage[TS1,T1]{fontenc}
\usepackage{array, booktabs}
\usepackage{graphicx}
\usepackage[x11names]{xcolor}
\usepackage{colortbl}
\usepackage{caption}
\DeclareCaptionFont{blue}{\color{LightSteelBlue3}}
\newcommand{\foo}{\color{LightSteelBlue3}\makebox[0pt]{\textbullet}\hskip-0.5pt\vrule width 1pt\hspace{\labelsep}}
% END TIMELINE

% Set the margins
\setlength{\textheight}{8.5in}
\setlength{\headheight}{.25in}
\setlength{\headsep}{.25in}
\setlength{\topmargin}{0in}
\setlength{\textwidth}{6.5in}
\setlength{\oddsidemargin}{0in}
\setlength{\evensidemargin}{0in}

% Set table floating
% use \begin{table}[H] to fix its position.
\floatstyle{plaintop}
\restylefloat{table}

%%%%%%%%%%%%%%%%%%%%%%%%%%%%%%%%%%%%%%%%%%%%%%%%%%%%%%%%%%%%%%%%%%%%%%%
% Macros

% Math Macros.  It would be better to use the AMS LaTeX package,
% including the Bbb fonts, but I'm showing how to get by with the most
% primitive version of LaTeX.  I follow the naming convention to begin
% user-defined macro and variable names with the prefix "my" to make it
% easier to distiguish user-defined macros from LaTeX commands.
%
\newcommand{\myfunction}[3] {${#1} : {#2} \rightarrow {#3}$ }
\newcommand{\myzrfunction}[1] {\myfunction{#1}{{\myZ}}{{\myR}}}
\renewcommand{\iff}{\Leftrightarrow}



% Formating Macros %

% header
\newcommand{\myheader}[4] {\vspace*{-0.5in} \noindent{#1} \hfill {#2} \newline \noindent{#3} \hfill {#4} \noindent \rule[8pt]{\textwidth}{1pt} \vspace{1ex} }
\newcommand{\myalgsheader}[0] {\myheader{MASc Research Weekly Update}{UTIAS SRG}{Anton Rubisov}{\today}} 
% end header

\newcommand\undermat[2]{% http://tex.stackexchange.com/a/102468/5764
  \makebox[0pt][l]{$\smash{\underbrace{\phantom{%
    \begin{smallmatrix}#2\end{smallmatrix}}}_{\text{$#1$}}}$}#2}
    
\newcommand\overmat[2]{%
  \makebox[0pt][l]{$\smash{\overbrace{\phantom{%
    \begin{smallmatrix}#2\end{smallmatrix}}}^{\text{$#1$}}}$}#2}
    
    
\newcommand\mat[1]{\boldsymbol{#1}}


%%%%%% Begin document with header and title %%%%%%%%%%%%%%%%%%%%%%%%%

\begin{document}
\myalgsheader
\pagestyle{plain}
{\begin{center} {\large {\bf Limit Order Book Dynamics}} \end{center}}
\bigskip

%%%%% Begin body %%%%%%%%%%%%%%%%%%%%%%%%%%%%%%%%%%%%%%%%%%%%%%%%%%%

\begin{quote}
Our goal is to use the dynamics of the Limit Order Book (LOB) as an indicator for
high-frequency stock price movement, thus enabling statistical arbitrage. Formally, we will the study limit order book imbalance process, $I(t)$, and the stock price process, $S(t)$, and attempt to establish a stochastic relationship $\dot{S} = f(S,I,t)$. We will then attempt to derive an optimal trading strategy based on the observed relationship.
\end{quote}

\section*{Roadmap and Timeline of Next Steps}

\begin{table}[H]
\renewcommand\arraystretch{1.4}\arrayrulecolor{LightSteelBlue3}
\captionsetup{singlelinecheck=false, font=blue, labelfont=sc, labelsep=quad}
\caption{Timeline}\vskip -1.5ex
\begin{tabular}{@{\,}r <{\hskip 2pt} !{\foo} >{\raggedright\arraybackslash}p{10cm}}
\toprule
\addlinespace[1.5ex]
end of 2014 & Complete CTMC calibration \\
end of 2014 & Backtest naive strategies based on CTMC \\
Jan-Feb 2015 & Study stochastic controls and MDPs / POMDPs, specifically in ECE1639H or STA2006H \\
Feb 2015 & Formulate the LOB dynamics and trade strategy in the stochastic control framework \\
Mar 2015 & Use dynamic programming to solve the stochastic control problem \\
Mar 2015 & Backtest optimal controller as trading strategy; compare with naive strategy \\
Apr-May 2015 & Revisit HMM to model the underlying dynamics \\
? & Write thesis report
\end{tabular}
\end{table}



%%% End solution and document %%%%%%%%%%%%%%%%%%%%%%%%%%%%%%%%%%%%

\end{document}
