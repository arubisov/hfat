% LaTeX set-up adapted from a template by Alan T. Sherman (9/9/98)
%%%%%%%%%%%%%%%%%%%%%%%%%%%%%%%%%%%%%%%%%%%%%%%%%%%%%%%%%%%%%%%%%%%%%%
\documentclass[12pt]{article}
\input{/home/anton/Documents/latex/LaTeXHeader.tex} % put hard path to header.
\usepackage{soul}
\newtheorem{theorem}{Theorem}
\usepackage{hyperref}

%%%%%% Begin document with header and title %%%%%%%%%%%%%%%%%%%%%%%%%

\begin{document}
\mymascheader
\pagestyle{plain}
{\begin{center} {\large {\bf High-Frequency Algorithmic Trading \\ with Momentum and Order Imbalance}} \end{center}}
\bigskip

%%%%% Begin body %%%%%%%%%%%%%%%%%%%%%%%%%%%%%%%%%%%%%%%%%%%%%%%%%%%

\begin{quote}
My goal is to establish and solve the stochastic optimal control problem that 
captures the momentum and order imbalance dynamics of the Limit Order Book 
(LOB). The solution will yield an optimal trading strategy that will permit
statistical arbitrage of the underlying stock, which will then be backtested on
historical data.
\end{quote}

\section*{Progress Timeline}
\begin{table}[H]
\renewcommand\arraystretch{1.4}\arrayrulecolor{LightSteelBlue3}
\newcommand{\foo}{\color{LightSteelBlue3}\makebox[0pt]{\textbullet}\hskip-0.5pt\vrule width 1pt\hspace{\labelsep}}
\newcommand{\fooo}{\color{LightSteelBlue3}\hskip-0.5pt\vrule width 1pt\hspace{\labelsep}}
\begin{tabular}{@{\,}r <{\hskip 2pt} !{\foo} >{\raggedright\arraybackslash}p{9cm} !{\fooo} >{\raggedright\arraybackslash}p{5cm}} 
\multicolumn{1}{@{\,}r <{\hskip 9pt}}{DATE} & \multicolumn{1}{l}{THESIS} & \multicolumn{1}{l}{STA4505} \\
\hline
\st{Dec 2014} & \st{Complete CTMC calibration} \\
\st{Dec 2014} & \st{Backtest naive strategies based on CTMC} & \\
\st{Jan-May} & \st{Study stochastic controls: ECE1639, STA4505} & \\
\st{Jun 5} & \st{Establish models} & \st{Exam Study} \\
\st{Jun 12} & \st{Establish performance criteria} & \st{Exam Study} \\
\st{Jun 15} & \st{Derive DPP/DPE} & \st{EXAM} \\
\st{Jun 19} & \st{Derive DPP/DPE} &  \\
\st{Jun 26} & \st{Derive continuous-time equations} & \\
\st{Jul 3} & \textcolor{red}{Derive discrete-time equations} & \\
Jul 10 & Set up MATLAB numerical integration &  \\
Jul 17 & Integrate functions and plot dynamics & Integrate and analyze too! \\
Jul 24 & More dynamics, and calib/choose parameters & \\
Jul 31 & Backtest on historical data & Simulate results \\
Aug 7 & More backtesting, comparing with previous & \\
Aug 14 & Dissertation writeup / buffer & Project writeup \\
Aug 21 & Dissertation writeup / buffer &  \\
Aug 28 & Dissertation writeup & Presentation \\
\end{tabular}
\end{table}

\newpage
\section*{Whiteboard Inspirational Quote of the Week}
\begin{quote}
\textit{What you do speaks so loudly that I cannot hear what you say.} \\- Ralph Waldo Emerson
\end{quote} 

\section*{For Our Readers in the Middle East...}
\url{http://52weeksofrejection.com}

This morning I was officially out of protein powder. It's amazing because it's the sort of thing you see coming form a mile away, like it's a very steady linear decline in supply so it's really never a surprise. And yet almost every time, I don't have the resolve to order a resupply until \textit{after} I suffer at least one juice-free disappointing day. Don't worry, banana cream pie is on its way. But so today I decided I'd make the morning smoothie and substitute a green apple instead of the protein powder, having realized a long time ago that the major role of the powder isn't the nutritional punch but the delicate dousing of deliciousness. But uh ... yeah never again.

Old Faithful is on life support. Multi-tabbing in Chrome is almost out of the cards. I don't really know what to do. I want a nice lightweight, 15-17", Linux-friendly machine as a replacement but I also have no disposable income anymore and I'm still somehow racking up monthly \$2000 visa bills. 

Look man you asked me to record, in the moment, why I was doing it. Here's why: deep down I've got cosmic knowledge that this one isn't over, and I'm positive that if I let it go now then it WILL be. Maybe it's doomed due to timing as we said, and if so then I guess I'll know pretty soon, but regardless, what I'm doing now \textit{does} seem like throwing everything away, and fuck that, that's not how we roll, we bury shit in the sand for a first date, you know what I'm saying? And in case you don't, I guess what I'm saying is that until it's 52-weeks worthy (ok, irony that it already did make it into 52weeks isn't lost on me...) I ain't just giving up.

\section*{The Academic Week in Review}
We are trying to solve the following DPE:
\begin{align}
\begin{split}
0 & = \max \Biggl\lbrace 
%%% Only Limit Orders
\sup_{\delta^\pm} \biggl\lbrace \E_{\bw} \biggl[
(s + \pi + \delta^-)L_k^- - (s - \pi - \delta^+)L_k^+ \\
& \hphantom{\sup_{\bu} \biggl\lbrace \E_{\bw} \biggl[ {}+{}} + (L_k^+ - L_k^-) \bigl( s + \eta_{0,\bz} T(\bz, \omega)^{(2)}  -\sgn( q + L_k^+ - L_k^-)\pi   \bigr) \\
& \hphantom{\sup_{\bu} \biggl\lbrace \E_{\bw} \biggl[ {}+{}} + q\left( \eta_{0,\bz} T(\bz, \omega)^{(2)}  -\left( \sgn( q + L_k^+ - L_k^-) - \sgn(q) \right) \pi \right) \\
& \hphantom{\sup_{\bu} \biggl\lbrace \E_{\bw} \biggl[ {}+{}} + h_{k+1}(T(\bz,\omega), q + L_k^+ - L_k^- ) -  h_k(\bz,q) \biggr] \biggr\rbrace \; ;\\
%%% Limit Orders + Market Buy
& \hphantom{{}={} \max \biggl\lbrace } \sup_{\delta^\pm} \biggl\lbrace \E_{\bw} \biggl[
(s + \pi + \delta^-)L_k^- - (s - \pi - \delta^+)L_k^+ - (s + \pi)\\
& \hphantom{\sup_{\bu} \biggl\lbrace \E_{\bw} \biggl[ {}+{}} + (L_k^+ - L_k^- +1) \bigl( s + \eta_{0,\bz} T(\bz, \omega)^{(2)}  -\sgn( q + L_k^+ - L_k^- +1)\pi   \bigr) \\
& \hphantom{\sup_{\bu} \biggl\lbrace \E_{\bw} \biggl[ {}+{}} + q\left( \eta_{0,\bz} T(\bz, \omega)^{(2)}  -\left( \sgn( q + L_k^+ - L_k^- +1) - \sgn(q) \right) \pi \right) \\
& \hphantom{\sup_{\bu} \biggl\lbrace \E_{\bw} \biggl[ {}+{}} + h_{k+1}(T(\bz,\omega), q + L_k^+ - L_k^- + 1) -  h_k(\bz,q) \biggr] \biggr\rbrace \; ; \\
%%% Limit Orders + Market Sell
& \hphantom{{}={} \max \biggl\lbrace } \sup_{\delta^\pm} \biggl\lbrace \E_{\bw} \biggl[
(s + \pi + \delta^-)L_k^- - (s - \pi - \delta^+)L_k^+ + (s - \pi) \\
& \hphantom{\sup_{\bu} \biggl\lbrace \E_{\bw} \biggl[ {}+{}} + (L_k^+ - L_k^- -1) \bigl( s + \eta_{0,\bz} T(\bz, \omega)^{(2)}  -\sgn( q + L_k^+ - L_k^- -1)\pi   \bigr) \\
& \hphantom{\sup_{\bu} \biggl\lbrace \E_{\bw} \biggl[ {}+{}} + q\left( \eta_{0,\bz} T(\bz, \omega)^{(2)}  -\left( \sgn( q + L_k^+ - L_k^- -1) - \sgn(q) \right) \pi \right) \\
& \hphantom{\sup_{\bu} \biggl\lbrace \E_{\bw} \biggl[ {}+{}} + h_{k+1}(T(\bz,\omega), q + L_k^+ - L_k^- -1) -  h_k(\bz,q) \biggr] \biggr\rbrace  \Biggr\rbrace
\end{split}
\end{align}

We'll begin by concentrating on the first case where $M^+ = M^- = 0$. Thus, we want to solve
\begin{equation}
\label{eq:discretesup1}
\begin{split}
& \sup_{\delta^\pm} \biggl\lbrace \E_{\bw} \biggl[
(s + \pi + \delta^-)L_k^- - (s - \pi - \delta^+)L_k^+ \\
& \hphantom{\sup_{\bu} \biggl\lbrace \E_{\bw} \biggl[ {}+{}} + (L_k^+ - L_k^-) \bigl( s + \eta_{0,\bz} T(\bz, \omega)^{(2)}  -\sgn( q + L_k^+ - L_k^-)\pi   \bigr) \\
& \hphantom{\sup_{\bu} \biggl\lbrace \E_{\bw} \biggl[ {}+{}} + q\left( \eta_{0,\bz} T(\bz, \omega)^{(2)}  -\left( \sgn( q + L_k^+ - L_k^-) - \sgn(q) \right) \pi \right) \\
& \hphantom{\sup_{\bu} \biggl\lbrace \E_{\bw} \biggl[ {}+{}} + h_{k+1}(T(\bz,\omega), q + L_k^+ - L_k^- ) -  h_k(\bz,q) \biggr] \biggr\rbrace
\end{split}
\end{equation}

First, some preliminary results:
\begin{align*}
\P [ L_k^- = 1 | K_k^+ = 1] & = e^{-\kappa \delta^-} \\
\P [ L_k^- = 0 | K_k^+ = 1] & = 1 - e^{-\kappa \delta^-} \\
\P [ L_k^- = 0 | K_k^+ = n] & = (1 - e^{-\kappa \delta^-})^n \\
\P [ L_k^- = 1 | K_k^+ = n] & = 1 - (1 - e^{-\kappa \delta^-})^n \\
\P [ K_k^+ = n] & = \frac{e^{-\lambda \Delta t} (\lambda \Delta t)^n}{n!} \\
\end{align*}

\begin{theorem}
$\E [ L_k^- ] = \P [ L_k^- = 1] =  1 - e^{-e^{-\kappa \delta^-} \lambda \Delta t}$
\end{theorem}
\begin{proof}
\begin{align*}
\E [ L_k^- ] & = \sum_{n=0}^\infty \P [ L_k^- = 1 | K_k^+ = n] \cdot \P[K_k^+ = n] \\
& =  \sum_{n=0}^\infty \bigl[ 1 - (1 - e^{-\kappa \delta^-})^n \bigr] \frac{e^{-\lambda \Delta t} (\lambda \Delta t)^n}{n!} \\
& = 1 - e^{-\lambda \Delta t} \sum_{n=0}^\infty \frac{(1 - e^{-\kappa \delta^-})^n (\lambda \Delta t)^n}{n!} \\
& = 1 - e^{-\lambda \Delta t} \sum_{n=0}^\infty \frac{(\lambda \Delta t - e^{-\kappa \delta^-}\lambda \Delta t)^n}{n!} \\
& = 1 - e^{-\lambda \Delta t} e^{\lambda \Delta t - e^{-\kappa \delta^-} \lambda \Delta t} \\
& = 1 - e^{- e^{-\kappa \delta^-} \lambda \Delta t}
\end{align*}
\end{proof}

For ease of notation, we'll write the probability of the $L_k^- = 0$ event as
\[ p(\delta^-) = e^{- e^{-\kappa \delta^-} \lambda \Delta t} \]
and its derivative as
\[ d(\delta^-) = \partial_{\delta^-} p(\delta^-) = \kappa \lambda \Delta t e^{-e^{-\kappa \delta^-} \lambda \Delta t - \kappa \delta^-} \]

This gives us the results:
\begin{align*}
\P [ L_k^- = 1] & = 1 - p(\delta^-) = \E [ L_k^-] \\
\P [ L_k^- = 0] & = p(\delta^-) \\
\partial_{\delta^-} \P [ L_k^- = 1]  & = -d(\delta^-) \\
\partial_{\delta^-} \P [ L_k^- = 0] & = d(\delta^-) \\
\end{align*}

Let's pre-compute some of the terms that we'll encounter in the supremum, namely the expectations of the random variables. 

\begin{align}
\E [\sgn(q + L_k^+ - L_k^-)] & = \P[L_k^- = 1] \cdot \P[L_k^+ = 1] \cdot \sgn(q) \nonumber \\
& \quad + \P[L_k^- = 1] \cdot \P[L_k^+ = 0] \cdot \sgn(q - 1) \nonumber \\
& \quad +  \P[L_k^- = 0] \cdot \P[L_k^+ = 1] \cdot \sgn(q+1) \nonumber \\
& \quad + \P[L_k^- = 0] \cdot \P[L_k^+ = 0] \cdot \sgn(q) \nonumber \\
& = (1 - p(\delta^-))(1-p(\delta^+)) \sgn(q) \nonumber \\
& \quad + (1 - p(\delta^-)) p(\delta^+) \sgn(q - 1) \nonumber \\
& \quad + p(\delta^-) (1-p(\delta^+))  \sgn(q+1) \nonumber \\
& \quad + p(\delta^-) p(\delta^+)  \sgn(q) \nonumber \\
& = \sgn(q) \bigl[ 1 - p(\delta^+) - p(\delta^-) + 2 p(\delta^+) p(\delta^-) \bigr] \nonumber \\
& \quad + \sgn(q-1) \bigl[ p(\delta^+)  - p(\delta^+) p(\delta^-) \bigr]  \nonumber \\
& \quad + \sgn(q+1) \bigl[ p(\delta^-)  - p(\delta^+) p(\delta^-) \bigr]  \nonumber \\
& = \begin{cases} 
1 & q \geq 2 \\
1 - p(\delta^+)(1 - p(\delta^-)) & q = 1 \\
p(\delta^-) - p(\delta^+) & q = 0 \\
-\bigl[ 1 - p(\delta^-)(1 - p(\delta^+)) \bigr] & q = -1 \\
-1 & q \leq -2
\end{cases} \\
& = \Phi(q, \delta^+, \delta^-)
\end{align}
Hence, we can also compute the partial derivatives of this expectation:
\begin{align}
\partial_{\delta^-} \Phi(q, \delta^+, \delta^-) & =  \begin{cases} 
0 & q \geq 2 \\
d(\delta^-)p(\delta^+) & q = 1 \\
d(\delta^-) & q = 0 \\
d(\delta^-)(1 - p(\delta^+))  & q = -1 \\
0 & q \leq -2
\end{cases} \\
\partial_{\delta^+} \Phi(q, \delta^+, \delta^-) & =  \begin{cases} 
0 & q \geq 2 \\
- d(\delta^+)(1 - p(\delta^-)) & q = 1 \\
- d(\delta^+) & q = 0 \\
- d(\delta^+) p(\delta^-) & q = -1 \\
0 & q \leq -2
\end{cases}
\end{align}
As it'll be required for later, we'll additionally introduce the shorthand notation
\begin{equation}
\frac{\partial_{\delta^-} \Phi(q, \delta^+, \delta^-)}{d(\delta^-)} = \Psi(q,\delta^+)
\end{equation}
\begin{equation}
\frac{\partial_{\delta^+} \Phi(q, \delta^+, \delta^-)}{d(\delta^+)} = \Upsilon(q,\delta^-)
\end{equation}
If we call $\bP$ the transition matrix for the Markov Chain $\bZ$, with $\bP_{\bz, \bj} = \P[\bZ_{k+1} = \bj | \bZ_k = \bz]$, then we can also write:
\begin{equation}
\begin{split}
\E[h_{k+1}(T(\bz,\omega), q + L_k^+ - L_k^- )] & = \sum_\bj \bP_{\bz,\bj} \biggl[ h_{k+1}(\bj, q) \bigl[ 1 - p(\delta^+) - p(\delta^-) + 2 p(\delta^+) p(\delta^-) \bigr]  \\
& \hphantom{\sum_\bj \bP_{\bz,\bj} \biggl[ {}+{}} + h_{k+1}(\bj, q-1) \bigl[ p(\delta^+)  - p(\delta^+) p(\delta^-) \bigr]   \\
& \hphantom{\sum_\bj \bP_{\bz,\bj} \biggl[ {}+{}} + h_{k+1}(\bj, q+1) \bigl[ p(\delta^-)  - p(\delta^+) p(\delta^-) \bigr] \biggr]  \\
\end{split}
\end{equation}
and its partial derivatives as
\begin{align}
\begin{split}
\partial_{\delta^-} \E[h_{k+1}(T(\bz,\omega), q + L_k^+ - L_k^- )] & = 
\sum_\bj \bP_{\bz,\bj} \biggl[ h_{k+1}(\bj, q) \bigl[ - d(\delta^-) + 2 p(\delta^+) d(\delta^-) \bigr]  \\
& \hphantom{\sum_\bj \bP_{\bz,\bj} \biggl[ {}+{}} + h_{k+1}(\bj, q-1) \bigl[  - p(\delta^+) d(\delta^-) \bigr]   \\
& \hphantom{\sum_\bj \bP_{\bz,\bj} \biggl[ {}+{}} + h_{k+1}(\bj, q+1) \bigl[ d(\delta^-)  - p(\delta^+) d(\delta^-) \bigr] \biggr]
\end{split} \\
\begin{split}
& = d(\delta^-) \sum_\bj \bP_{\bz,\bj} \biggl[ h_{k+1}(\bj, q) \bigl[ - 1 + 2 p(\delta^+) \bigr]  \\
& \hphantom{d(\delta^-) \sum_\bj \bP_{\bz,\bj} \biggl[ {}+{}} + h_{k+1}(\bj, q-1) \bigl[  - p(\delta^+) \bigr]   \\
& \hphantom{d(\delta^-) \sum_\bj \bP_{\bz,\bj} \biggl[ {}+{}} + h_{k+1}(\bj, q+1) \bigl[ 1  - p(\delta^+) \bigr] \biggr]
\end{split} \\
\begin{split}
\partial_{\delta^+} \E[h_{k+1}(T(\bz,\omega), q + L_k^+ - L_k^- )] & = 
d(\delta^+) \sum_\bj \bP_{\bz,\bj} \biggl[ h_{k+1}(\bj, q) \bigl[ - 1 + 2 p(\delta^-) \bigr]  \\
& \hphantom{d(\delta^-) \sum_\bj \bP_{\bz,\bj} \biggl[ {}+{}} + h_{k+1}(\bj, q-1) \bigl[1  - p(\delta^-) \bigr]   \\
& \hphantom{d(\delta^-) \sum_\bj \bP_{\bz,\bj} \biggl[ {}+{}} + h_{k+1}(\bj, q+1) \bigl[  - p(\delta^-) \bigr] \biggr]
\end{split}
\end{align}

Now we tackle solving the supremum in equation $\ref{eq:discretesup1}$. First we consider the first-order condition on $\delta^-$, namely that the partial derivative with respect to it must be equal to zero.
\begin{align}
\begin{split}
0 & = (1-p(\delta^-)) - d(\delta^-)(s+\pi+\delta^-) + d(\delta^-)\bigr(s + \E[\eta_{0,\bz}T(\bz,\omega)^{(2)}] - \Phi(q,\delta^+,\delta^-)\pi \bigl) \\
& \quad -\partial_{\delta^-}\Phi(q,\delta^+,\delta^-)\pi \bigl( p(\delta^-) - p(\delta^+) \bigr) - q \pi \partial_{\delta^-}\Phi(q,\delta^+,\delta^-) + d(\delta^-) \sum_\bj \bP_{\bz,\bj} \\
& \quad \times \biggl[ h_{k+1}(\bj, q) \bigl[ - 1 + 2 p(\delta^+) \bigr]  + h_{k+1}(\bj, q-1) \bigl[  - p(\delta^+) \bigr] + h_{k+1}(\bj, q+1) \bigl[ 1  - p(\delta^+) \bigr] \biggr]
\end{split}
\intertext{and diving through by $d(\delta^-)$, which is nonzero, we get}
\begin{split}
& = \frac{1-p(\delta^-)}{d(\delta^-)} -(s+\pi+\delta^-) + \bigr(s + \E[\eta_{0,\bz}T(\bz,\omega)^{(2)}] - \Phi(q,\delta^+,\delta^-)\pi \bigl) \\
& \quad -\pi \bigl( p(\delta^-) - p(\delta^+) \bigr)\Psi(q,\delta^+) - q \pi\Psi(q,\delta^+)\\
& \quad + \sum_\bj \bP_{\bz,\bj} \biggl[ h_{k+1}(\bj, q) \bigl[ - 1 + 2 p(\delta^+) \bigr]   + h_{k+1}(\bj, q-1) \bigl[  - p(\delta^+) \bigr] + h_{k+1}(\bj, q+1) \bigl[ 1  - p(\delta^+) \bigr] \biggr]
\end{split}
\end{align}


Re-arranging and collecting terms in $\delta^-$ on the left-hand side:
\begin{equation}
\begin{split}
& \frac{1-p(\delta^-)}{d(\delta^-)} + \delta^- + \pi \Phi(q,\delta^+,\delta^-) + \pi p(\delta^-)\Psi(q,\delta^+) \\
& = \E[\eta_{0,\bz}T(\bz,\omega)^{(2)}] + \pi\left( -1 + p(\delta^+)\Psi(q,\delta^+) - q\Psi(q,\delta^+)\right) \\
& + \sum_\bj \bP_{\bz,\bj} \biggl[ h_{k+1}(\bj, q) \bigl[ - 1 + 2 p(\delta^+) \bigr]   + h_{k+1}(\bj, q-1) \bigl[  - p(\delta^+) \bigr] + h_{k+1}(\bj, q+1) \bigl[ 1  - p(\delta^+) \bigr] \biggr]
\end{split}
\end{equation}

\[ \frac{1-p(\delta^-)}{d(\delta^-)} = \frac{e^{\kappa \delta^-}\left( e^{\lambda \Delta t e^{-\kappa \delta^-}} \right)}{\kappa\lambda\Delta t} \]

\section*{Looking Ahead}
Look, this week has been a bit of a bust for a variety of reasons. Academically speaking, it's because I'm fucking stuck. I don't see where these derivations are going, because they're just fucking messy. When I finally came to this conclusion I immediately reached out to both Sebastian and Gabe but only the latter wrote back and said he'd look at it. I know, I've gotta be more pushy. I'll follow up with Seb again tomorrow. 

There's good news, though. Realistically I don't need to do all this linearly, and I can begin looking at the numerical integration part that's due next week anyway. In fact, yes, precisely. I can just shelf this till I can get either of their help and move the fuck on. Efficiency, bitches! Equally in the good news department is that I already did a couple hours of discussion with my dad on the topic of numerical solutions, as it turns out I've never \textit{actually} done it, although I guess in a very true sense Monte Carlo sims are exactly that. Anyway, word on the street is that you can use virtually whatever fucking method you want, we've got a few contenders, probably will go with the most complicated one to get my money's worth from these 8 CPU's.
\end{document}
