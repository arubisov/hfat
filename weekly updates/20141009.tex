% LaTeX set-up adapted from a template by Alan T. Sherman (9/9/98)
%%%%%%%%%%%%%%%%%%%%%%%%%%%%%%%%%%%%%%%%%%%%%%%%%%%%%%%%%%%%%%%%%%%%%%
\documentclass[12pt]{article}
\usepackage{amsmath}
\usepackage{amsfonts}
\usepackage{tabularx}
\usepackage{float}

% Set the margins
%
\setlength{\textheight}{8.5in}
\setlength{\headheight}{.25in}
\setlength{\headsep}{.25in}
\setlength{\topmargin}{0in}
\setlength{\textwidth}{6.5in}
\setlength{\oddsidemargin}{0in}
\setlength{\evensidemargin}{0in}

% Set table floating
% use \begin{table}[H] to fix its position.
\restylefloat{table}

%%%%%%%%%%%%%%%%%%%%%%%%%%%%%%%%%%%%%%%%%%%%%%%%%%%%%%%%%%%%%%%%%%%%%%%
% Macros

% Math Macros.  It would be better to use the AMS LaTeX package,
% including the Bbb fonts, but I'm showing how to get by with the most
% primitive version of LaTeX.  I follow the naming convention to begin
% user-defined macro and variable names with the prefix "my" to make it
% easier to distiguish user-defined macros from LaTeX commands.
%
\newcommand{\myfunction}[3] {${#1} : {#2} \rightarrow {#3}$ }
\newcommand{\myzrfunction}[1] {\myfunction{#1}{{\myZ}}{{\myR}}}
\renewcommand{\iff}{\Leftrightarrow}

% Formating Macros %

\newcommand{\myheader}[4] {\vspace*{-0.5in} \noindent{#1} \hfill {#2} \newline \noindent{#3} \hfill {#4} \noindent \rule[8pt]{\textwidth}{1pt} \vspace{1ex} }  % end \myheader 

\newcommand{\myalgsheader}[0] {\myheader{MASc Research Weekly Update}{UTIAS SRG}{Anton Rubisov}{\today}} 

% Running head (goes at top of each page, beginning with page 2.
% Must precede by \pagestyle{myheadings}.
\newcommand{\myrunninghead}[2] {\markright{{\it {#1}, {#2}}}}

\newcommand{\myrunningalgshead}[2] {\myrunninghead{MAT315S Number Theory}{{#1}}}

\newcommand{\myrunninghwhead}[1] {\myrunningalgshead{Solutions to Tutorial Problems {#1}}}

\newcommand{\mytitle}[1] {\begin{center} {\large {\bf {#1}}} \end{center}}

\newcommand{\myhwtitle}[1] {\begin{center} {\large {\bf Limit Order Book Dynamics}} \end{center}}

\newcommand{\mysection}[1] {\noindent {\bf {#1}}}

\newcommand\mat[1]{\boldsymbol{#1}}

%%%%%% Begin document with header and title %%%%%%%%%%%%%%%%%%%%%%%%%

\begin{document}
\myalgsheader
\pagestyle{plain}
\myhwtitle{1}
\bigskip

%%%%% Begin body %%%%%%%%%%%%%%%%%%%%%%%%%%%%%%%%%%%%%%%%%%%%%%%%%%%

\begin{quote}
Our goal is to use the dynamics of the Limit Order Book (LOB) as an indicator for
high-frequency stock price movement, thus enabling statistical arbitrage. Formally, we will the study limit order book imbalance process, $I(t)$, and the stock price process, $S(t)$, and attempt to establish a stochastic relationship $\dot{S} = f(S,I,t)$.
\end{quote}

\section*{Recap Next Steps}
\begin{enumerate}
\item Run cross-validation on the old CTMC imbalance model, also varying the averaging time.
\item Check for a unit root in the imbalance time series using the augmented Dickey-Fuller test, after transforming the data using the logit function.
\item Instead of running a HMM where the hidden state informs the observable imbalance, try having the hidden state affect the transition matrix between imbalance states. 
\item Consider a CTMC where the state is actually the pair $(I_{k-1}, I_k)$, with a $k^2 \times k^2$ transition matrix. Cross-validate and compare with regular CTMC.
\item Same as above but with HMM. 
\end{enumerate}

\section*{Cross-validation of CTMC}
To cross-validate the CTMC calibration, the following steps were taken:
\begin{enumerate}
\item An imbalance averaging time (in ms) and number of imbalance bins were fixed. The infinitesimal generator matrix $\mat{G}$ was calculated on the resulting timeseries. 
\item An embedded discrete Markov chain transition matrix $\mat{A}$ was obtained from $\mat{G}$. This effectively says: conditional on a transition from bin $i$, what are the transition probabilities to bin $j$?
\item The stationary distribution, and number ($n$) of steps required to converge to the stationary distribution, was calculated. That is: for $\epsilon > 0$, calculate $n$ such that $||\mat{A}^{n+1} - \mat{A}^n|| < \epsilon$.
\item Find the average number of steps in the timeseries that are required to observe $n$ transitions. This is the size of the timewindow against which to cross-validate. 
\item Remove the cross-validation timewindow (call this the ``removed series'') from the full timeseries (call this the ``remaining series''). Calculate two infinitesimal generator matrices $\mat{G}_{removed}$ and $\mat{G}_{remaining}$.
\item Calculate two error terms for the resulting matrices:
$$err = \sqrt{\dfrac{1}{\#trials} \times \sum\limits_{trials} \left( \dfrac{1}{\#bins^2} \sum\limits_{ij} (\mat{G}_{remaining}(ij) - \mat{G}_{removed}(ij))^2 \right)x} $$
$$\mat{Err} = \sqrt{\dfrac{1}{\#trials} \times \sum\limits_{trials} (\mat{1} - \mat{G}_{removed} \div \mat{G}_{remaining})^2 }$$
where, for $\mat{Err}$, division and squaring are entry-wise and not matrix-wise.
\end{enumerate}

\begin{table}[H]
\small
\centering
\vspace*{2.5mm}
\begin{tabularx} {\textwidth}{c|c|c|c|c}
\hline
\bf num bins \\ averaging time & stationary $n$ & Timewindow size & $err$ & $\mat{Err}$ \\
\hline\hline
3 bins, 100ms & 172 & 13629 steps (5.8\% of series) & 0.020345 & 50\% - 464\% \\
\hline
3 bins, 500ms & 150 & 2982 steps (6.4\% of series) & 0.012794 & 37\% - 331\% \\
\hline
3 bins, 1000ms & 120 & 1412 steps (6.0\% of series) & 0.010105 & 33\% - 264\% \\
\hline
3 bins, 2000ms & 121 & 866 steps (7.4\% of series) & 0.006786 & 27\% - 185\% \\
\hline
3 bins, 3000ms & 129 & 715 steps (9.2\% of series) & 0.005877 & 24\% - 214\% \\
\hline
3 bins, 5000ms & 114 & 497 steps (10.5\% of series) & 0.004026 & 15\% - 154\% \\
\hline
3 bins, 10000ms & 134 & 476 steps (20.3\% of series) & 0.001945 & 7\% - 84\% \\
\hline
3 bins, 20000ms & 167 & 492 steps (42\% of series) & 0.001326 & 5\% - 33\% \\
\hline
\hline
5 bins, 100ms & 53 & 2509 steps (1.1\% of series) & 0.140246 & 96\% - 7419\% \\
\hline
5 bins, 500ms & 46 & 554 steps (1.18\% of series) & 0.034655 & 66\% - 1035\% \\
\hline
5 bins, 1000ms & 40 & 289 steps (1.24\% of series) & 0.023179 & 63\% - 911\% \\
\hline
5 bins, 2000ms & 37 & 168 steps (1.44\% of series)  & 0.012056 & 52\% - 1441\% \\
\hline
5 bins, 3000ms & 38 & 137 steps (1.76\% of series) &  0.009366 & 44\% - Inf\% \\
\hline
5 bins, 5000ms & 32 & 93 steps (1.99\% of series) & 0.006778 & 41\% - 529\% \\
\hline
5 bins, 10000ms & 37 & 83 steps (3.55\% of series) & 0.003355 & 28\% - Inf\% \\
\hline
5 bins, 20000ms & 29 & 56 steps (4.79\% of series) & 0.002009 & 23\% - Inf\% \\
\hline
5 bins, 30000ms & 29 & 53 steps (6.79\% of series) & 0.001533 & 21\% - Inf\% \\
\hline
\end{tabularx}
\end{table}

\section*{HMM Calibration - Initial Guesses}
A source of variability in the HMM calibration was the random matrices provided as initial guessed for the Baum-Welch algorithm. (That is, the transition matrix $\mat{T}_{guess}$ and emission matrix $\mat{E}_{guess}$ were chosen to contain random numbers, and then subjected to the constraints of probability matrices, namely having rows sum to 1.) 

Instead,  $\mat{E}_{guess}$ was chosen to have all rows equal to the actual observed distribution of imbalance bins. So, for a 3-bin calibration, if the three bins were empirically observed to occur 15\%, 60\%, and 25\% of the time, then each row of $\mat{E}_{guess}$ would be $[ \begin{smallmatrix} 0.15 & 0.6 & 0.25 \end{smallmatrix} ]$.

Choosing a good guess for $\mat{T}_{guess}$ is substantially more difficult, and literature on the topic is non-existent. Intuitively, because the states are hidden, no empirical observations should inform the transition values. 

We tried the following choice of matrices: diagonal elements were chosen to be equal to a value between 0.9 and 1.0, and off-diagonal elements were all chosen equal such that rows summed to 1. Thus, an example matrix would be:

$$\mat{T}_{guess} = \begin{bmatrix} 0.98 & 0.01 & 0.01 \\ 0.01 & 0.98 & 0.01 \\ 0.01 & 0.01 &  0.98 \end{bmatrix}$$

We tested calibrating an HMM with 3 hidden states and 3 bins, with initial guesses for diagonal elements ranging from 0.9 to 0.999 in increments of 0.001. Most variation in calibration results was due to re-ordering of the hidden states. Up to reordering, then, the resulting calibrations were:

\begin{table}[H]
\small
\centering
\caption{10s imbalance averaging period.}%
\vspace*{2.5mm}
\begin{tabularx}{0.72 \textwidth}{c|c}
\hline
Diagonal 0.999-0.970 & Diagonal 0.969 - 0.900 \\
\hline\hline
$\mat{T}_{est} = \begin{bmatrix}
    0.6105 &   0.3579  &  0.0316 \\
    0.0972 &   0.7679  &  0.1349 \\
    0.0120 &   0.2307  &  0.7572 
\end{bmatrix}$
&
$\mat{T}_{est} = \begin{bmatrix}
    0.7565 &   0.2357  &  0.0078 \\
    0.1362 &   0.7843  &  0.0795 \\
    0.0211 &   0.3296  &  0.6493 
\end{bmatrix}$ 
\\
\phantom{0} & 
\\
$\mat{E}_{est} = \begin{bmatrix}
    0.0000 &   0.0779  &  0.9221 \\
    0.0181 &   0.9819  &  0.0000 \\
    0.9827 &   0.0147  &  0.0027 
\end{bmatrix}$
&
$\mat{E}_{est} = \begin{bmatrix}
    0.9803 &   0.0186  &  0.0011 \\
    0.0154 &   0.9626  &  0.0219 \\
    0.0000 &   0.0734  &  0.9266 
\end{bmatrix}$  \\
\hline 
\end{tabularx}
\end{table}

\begin{table}[H]
\small
\centering
\caption{1s imbalance averaging period.}%
\vspace*{2.5mm}
\begin{tabularx}{0.98\textwidth}{c|c|c}
\hline
Diagonal 0.999-0.941 & Diagonal 0.940 - 0.932 & Diagonal 0.931 - 0.900 \\
\hline\hline
$\mat{T}_{est} = \begin{bmatrix}
    0.9362 &   0.0595  &  0.0044 \\
    0.0412 &   0.9291  &  0.0297\\
    0.0073 &   0.0965  &  0.8962
\end{bmatrix}$
& 
(bogus results)
& 
$\mat{T}_{est} = \begin{bmatrix}
    0.9291 &   0.0412  &  0.0297\\
    0.0593 &   0.9363  &  0.0044\\
    0.0965 &   0.0073  &  0.8962
\end{bmatrix}$
\\
\phantom{0} & & 
\\
$\mat{E}_{est} = \begin{bmatrix}
    0.9943 &   0.0054  &  0.0003\\
    0.0038 &   0.9934  &  0.0028\\
    0.0012 &   0.0073  &  0.9915
\end{bmatrix}$
&

&
$\mat{E}_{est} = \begin{bmatrix}
    0.0038 &   0.9934   & 0.0028\\
    0.9943 &   0.0054   & 0.0003\\
    0.0012 &   0.0073   & 0.9914
\end{bmatrix}$ \\
\hline 
\end{tabularx}
\end{table}

\section*{2-dimensional CTMC}
Next we considered a CTMC that tracks not only the imbalance bin, but jointly the imbalance bin and the price change over a subsequent interval. That is to say, the CTMC state was the pair $(I(t), S(t))$ where $I(t)$ is the bin corresponding to imbalance averaged over the interval $[t-\Delta t_I, t]$, and $S(t) = \text{sign}(S(t+\Delta t_S)-S(t))$. For 3 bins, this was encoded into one dimension $Z(t)$ as follows:

\begin{table}[H]
\small
\centering
\vspace*{2.5mm}
\begin{tabularx} {0.4\textwidth}{c|c|X}
\hline
$Z(t)$ & Bin $I(t)$ & Price change $S(t)$ \\
\hline\hline
1 & Bin 1 & $<0$ \\
\hline
2 & Bin 2 & $<0$ \\
\hline
3 & Bin 3 & $<0$ \\
\hline
4 & Bin 1 & $0$ \\
\hline
5 & Bin 2 & $0$ \\
\hline
6 & Bin 3 & $0$ \\
\hline
7 & Bin 1 & $>0$ \\
\hline
8 & Bin 2 & $>0$ \\
\hline
9 & Bin 3 & $>0$ \\
\hline
\end{tabularx}
\end{table}

Here bid and ask prices were considered separately rather than considering the change in mid price. Calibrating a CTMC on the two resulting timeseries $Z_{bid}(t)$ and $Z_{ask}(t)$ yielded some interesting results:


imbalance $\Delta t_I$: 1000ms, price $\Delta t_S$: 500ms

$$\mat{G}_{Z_{bid}} =
\begin{bmatrix}
-0.9928 &   0.0217   &      0 &   0.2826  &  0.5870  &  0.0870  &       0   & 0.0145     &    0 \\

    0.0118  & -0.9647   &     0  &  0.1412 &   0.5882 &   0.2000   &      0   & 0.0118  &  0.0118 \\
         0  &  0.0909  & -1.0000   &      0  &  0.3636  &  0.5455  &       0     & 0  &       0 \\
    0.0146  &  0.0005     &    0  & -0.0792  &  0.0562   & 0.0034 &   0.0036   & 0.0006  &  0.0003 \\
    0.0016 &   0.0052   & 0.0003 &   0.0435  & -0.0897 &   0.0300  &       0   & 0.0080 &   0.0011 \\
    0.0003&    0.0025 &   0.0022  &  0.0053   & 0.0919  & -0.1277     &    0   & 0.0017 &   0.0237 \\
         0  &  0.0345   &      0  &  0.4138  &  0.4138  &  0.1034 &  -1.0000   & 0.0345    &     0 \\
    0.0179   & 0.0179    &     0   & 0.2232   & 0.5536  &  0.1250 &   0.0089  & -0.9732 &   0.0268 \\
    0.0094&    0.0189   &      0  &  0.1132&    0.5189 &  0.3113    &     0   & 0.0094 &  -0.9811
\end{bmatrix}$$


$$\mat{G}_{Z_{ask}} =
\begin{bmatrix}
 -0.9915 &  0.0169 &  0 &  0.2881 &  0.5678 &  0.1017 &   0 &  0.0169 &  0 \\
 0.0106 & -0.9681 &  0 &  0.1277 &  0.5638 &  0.2340 &   0 &  0.0213 &  0.0106 \\
 0  &  0.0588 & -1.0000 &  0 &  0.2941 &  0.5882 &   0 &  0 &  0.0588 \\
 0.0121 &  0.0005 &  0 & -0.0775 &  0.0580 &  0.0034 &  0.0027 &  0.0005 &  0.0003 \\
 0.0016 &  0.0058 &  0.0002 &  0.0448 & -0.0898 &  0.0297 &   0 &  0.0065 &  0.0011 \\
 0.0003 &  0.0025 &  0.0039 &  0.0059 &  0.0907 & -0.1311 &   0 &  0.0008 &  0.0270 \\
 0 &  0.0476 &  0 &  0.1905 &  0.5714 &  0.1429 & -1.0000 &  0.0476 &   0 \\
 0 &  0.0440 &  0 &  0.1319 &  0.6374 &  0.1429 &   0 & -0.9890 &  0.0330 \\
 0.0085 &  0.0254 &  0.0085 &  0.0847 &  0.5169 &  0.3220 &   0 &  0.0169 & -0.9831

\end{bmatrix}$$

Using these matrices, we can compute conditional probabilities. For example, we can ask: conditional on being in bin 1 (more bid volume than ask) and on the bid price changing, what is the probability that the change will be greater than 0? less than 0?

Again, converting the generator matrix to the embedded discrete time Markov chain matrix proves enlightening for these calculations:

$$\mat{A}_{Z_{bid}} = 
\begin{bmatrix}
         0  &  0.0219 &        0 &   0.2847 &   0.5912 &   0.0876 &        0 &   0.0146 &        0\\
    0.0122  &       0 &        0 &   0.1463 &   0.6098 &   0.2073 &        0 &   0.0122 &   0.0122\\
         0  &  0.0909 &        0 &        0 &   0.3636 &   0.5455 &        0 &        0 &        0\\
    0.1839  &  0.0065 &        0 &        0 &   0.7097 &   0.0435 &   0.0452 &   0.0081 &   0.0032\\
    0.0174  &  0.0581 &   0.0029 &   0.4855 &        0 &   0.3343 &        0 &   0.0891 &   0.0126\\
    0.0022  &  0.0197 &   0.0175 &   0.0416 &   0.7199 &        0 &        0 &   0.0131 &   0.1860\\
         0  &  0.0345 &        0 &   0.4138 &   0.4138 &   0.1034 &        0 &   0.0345 &        0\\
    0.0183  &  0.0183 &        0 &   0.2294 &   0.5688 &   0.1284 &   0.0092 &        0 &   0.0275\\
    0.0096  &  0.0192 &        0 &   0.1154 &   0.5288 &   0.3173 &        0 &   0.0096 &        0
\end{bmatrix}$$

\begin{align*}
P(S(t+\delta t_S) > S(t) \; | \; S(t+\delta t_S) \neq S(t) \; \& \; I(t) = 1) 
&= \dfrac{ P(S(t+\delta t_S) > S(t) \; \& \; I(t) = 1)}{P(S(t+\delta t_S) \neq S(t) \; \& \; I(t) = 1)} \\
x & y
\end{align*}

%%% End solution and document %%%%%%%%%%%%%%%%%%%%%%%%%%%%%%%%%%%%

\end{document}
