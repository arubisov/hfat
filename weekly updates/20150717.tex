% LaTeX set-up adapted from a template by Alan T. Sherman (9/9/98)
%%%%%%%%%%%%%%%%%%%%%%%%%%%%%%%%%%%%%%%%%%%%%%%%%%%%%%%%%%%%%%%%%%%%%%
\documentclass[12pt]{article}
\input{/home/anton/Documents/latex/LaTeXHeader.tex} % put hard path to header.
\usepackage{soul}
\newtheorem{theorem}{Theorem}
\usepackage{hyperref}
\allowdisplaybreaks

%%%%%% Begin document with header and title %%%%%%%%%%%%%%%%%%%%%%%%%

\begin{document}
\mymascheader
\pagestyle{plain}
{\begin{center} {\large {\bf High-Frequency Algorithmic Trading \\ with Momentum and Order Imbalance}} \end{center}}
\bigskip

%%%%% Begin body %%%%%%%%%%%%%%%%%%%%%%%%%%%%%%%%%%%%%%%%%%%%%%%%%%%

\begin{quote}
My goal is to establish and solve the stochastic optimal control problem that 
captures the momentum and order imbalance dynamics of the Limit Order Book 
(LOB). The solution will yield an optimal trading strategy that will permit
statistical arbitrage of the underlying stock, which will then be backtested on
historical data.
\end{quote}

\section*{Progress Timeline}
\begin{table}[H]
\renewcommand\arraystretch{1.4}\arrayrulecolor{LightSteelBlue3}
\newcommand{\foo}{\color{LightSteelBlue3}\makebox[0pt]{\textbullet}\hskip-0.5pt\vrule width 1pt\hspace{\labelsep}}
\newcommand{\fooo}{\color{LightSteelBlue3}\hskip-0.5pt\vrule width 1pt\hspace{\labelsep}}
\begin{tabular}{@{\,}r <{\hskip 2pt} !{\foo} >{\raggedright\arraybackslash}p{9cm} !{\fooo} >{\raggedright\arraybackslash}p{5cm}} 
\multicolumn{1}{@{\,}r <{\hskip 9pt}}{DATE} & \multicolumn{1}{l}{THESIS} & \multicolumn{1}{l}{STA4505} \\
\hline
\st{Dec 2014} & \st{Complete CTMC calibration} \\
\st{Dec 2014} & \st{Backtest naive strategies based on CTMC} & \\
\st{Jan-May} & \st{Study stochastic controls: ECE1639, STA4505} & \\
\st{Jun 5} & \st{Establish models} & \st{Exam Study} \\
\st{Jun 12} & \st{Establish performance criteria} & \st{Exam Study} \\
\st{Jun 15} & \st{Derive DPP/DPE} & \st{EXAM} \\
\st{Jun 19} & \st{Derive DPP/DPE} &  \\
\st{Jun 26} & \st{Derive continuous-time equations} & \\
\st{Jul 3} & \st{Derive discrete-time equations} & \\
\st{Jul 10} & \st{Set up MATLAB numerical integration} &  \\
Jul 17 & Integrate functions and plot dynamics & \st{Integrate and analyze} \\
Jul 24 & More dynamics, and calib/choose parameters & \\
Jul 31 & Backtest on historical data & \st{Simulate results} \\
Aug 7 & More backtesting, comparing with previous & \\
Aug 14 & Dissertation writeup / buffer & Project writeup \\
Aug 21 & Dissertation writeup / buffer &  \\
Aug 28 & Dissertation writeup & Presentation \\
\end{tabular}
\end{table}

\newpage
\section*{Whiteboard Inspirational Quote of the Week}
\begin{quote}
\textit{Solving a problem for which you know there's an answer is like climbing a mountain with a guide, along a trail someone else has laid. In mathematics, the truth is somewhere out there in a place no one knows, beyond all the beaten paths. And it's not always at the top of the mountain. It might be in a crack on the smoothest cliff or somewhere deep in the valley.} \\-- Yoko Ogawa, \textit{The Housekeeper and the Professor}
\end{quote} 

\section*{For Our Readers in the Middle East...}


\section*{The Academic Week in Review}
This week we develop numerical solutions for the simplified dynamic programming equations, and thus solve for the unknown ansantz functions $h$. Generally speaking, we will use an explicit finite difference scheme to solve the differential equations, and a cutoff method for evaluating the the quasi-variational inequality. These two methods are explained in greater detail below.
\begin{equation}
\label{eq:discreteDPE}
\begin{split}
h_k(\bz,q) & = \max \Biggl\lbrace 
%%% Only Limit Orders
q\E[\eta_{0,\bz} T(\bz, \omega)^{(2)}] + \frac{1}{\kappa}(p({\delta^+}^*)+p({\delta^-}^*)) -2\pi p({\delta^+}^*)p({\delta^-}^*) \indicator_{q=0} \\ 
& \hphantom{{}={} \max \biggl\lbrace {}+{}} + p({\delta^+}^*)p({\delta^-}^*)\sum_\bj \bP_{\bz,\bj} \left[ h_{k+1}(\bj, q-1) + h_{k+1}(\bj, q+1) -2 h_{k+1}(\bj, q)  \right] \\
& \hphantom{{}={} \max \biggl\lbrace {}+{}} + \sum_{\bj} \bP_{\bz,\bj} h_{k+1}(\bj,q)  \; ;\\
%%% Limit Orders + Market Buy
& \hphantom{{}={} \max \biggl\lbrace} (q+1)\E[\eta_{0,\bz} T(\bz, \omega)^{(2)}] + \frac{1}{\kappa}(p({\delta_{2}^+}^*)+p({\delta_{2}^-}^*)) -2\pi p({\delta_{2}^+}^*)p({\delta_{2}^-}^*) \indicator_{q=-1} \\ 
& \hphantom{{}={} \max \biggl\lbrace {}+{}} + p({\delta_{2}^+}^*)p({\delta_{2}^-}^*)\sum_\bj \bP_{\bz,\bj} \left[ h_{k+1}(\bj, q) + h_{k+1}(\bj, q+2) -2 h_{k+1}(\bj, q+1)  \right] \\
& \hphantom{{}={} \max \biggl\lbrace {}+{}} + \sum_{\bj} \bP_{\bz,\bj} h_{k+1}(\bj,q+1) \; ;\\
%%% Limit Orders + Market Sell
& \hphantom{{}={} \max \biggl\lbrace} (q-1)\E[\eta_{0,\bz} T(\bz, \omega)^{(2)}] + \frac{1}{\kappa}(p({\delta_3^+}^*)+p({\delta_3^-}^*)) -2\pi p({\delta_3^+}^*)p({\delta_3^-}^*) \indicator_{q=1} \\ 
& \hphantom{{}={} \max \biggl\lbrace {}+{}} + p({\delta_3^+}^*)p({\delta_3^-}^*)\sum_\bj \bP_{\bz,\bj} \left[ h_{k+1}(\bj, q-2) + h_{k+1}(\bj, q) -2 h_{k+1}(\bj, q-1)  \right] \\
& \hphantom{{}={} \max \biggl\lbrace {}+{}} + \sum_{\bj} \bP_{\bz,\bj} h_{k+1}(\bj,q-1)  \Biggr\rbrace
\end{split}
\end{equation}

We now have an explicit means of numerically solving for the optimal posting depths. Since we know the function $h$ at the terminal timestep $T$, we can take one step back to $T-1$ and solve for each of the optimal posting depths in each of the cases. With these values we are then able to calculate the value function $h_{T-1}$, and in doing so determine whether to execute market orders in addition to posting limit orders (essentially taking the $\argmax$ instead of the $\max$ in equation \ref{eq:discreteDPE}). This process then repeats for each step backward.
\section*{Looking Ahead}

\end{document}
