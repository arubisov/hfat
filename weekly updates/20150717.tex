% LaTeX set-up adapted from a template by Alan T. Sherman (9/9/98)
%%%%%%%%%%%%%%%%%%%%%%%%%%%%%%%%%%%%%%%%%%%%%%%%%%%%%%%%%%%%%%%%%%%%%%
\documentclass[12pt]{article}
\input{/home/anton/Documents/latex/LaTeXHeader.tex} % put hard path to header.
\usepackage{soul}
\newtheorem{theorem}{Theorem}
\usepackage{hyperref}
\allowdisplaybreaks

%%%%%% Begin document with header and title %%%%%%%%%%%%%%%%%%%%%%%%%

\begin{document}
\mymascheader
\pagestyle{plain}
{\begin{center} {\large {\bf High-Frequency Algorithmic Trading \\ with Momentum and Order Imbalance}} \end{center}}
\bigskip

%%%%% Begin body %%%%%%%%%%%%%%%%%%%%%%%%%%%%%%%%%%%%%%%%%%%%%%%%%%%

\begin{quote}
My goal is to establish and solve the stochastic optimal control problem that 
captures the momentum and order imbalance dynamics of the Limit Order Book 
(LOB). The solution will yield an optimal trading strategy that will permit
statistical arbitrage of the underlying stock, which will then be backtested on
historical data.
\end{quote}

\section*{Progress Timeline}
\begin{table}[H]
\renewcommand\arraystretch{1.4}\arrayrulecolor{LightSteelBlue3}
\newcommand{\foo}{\color{LightSteelBlue3}\makebox[0pt]{\textbullet}\hskip-0.5pt\vrule width 1pt\hspace{\labelsep}}
\newcommand{\fooo}{\color{LightSteelBlue3}\hskip-0.5pt\vrule width 1pt\hspace{\labelsep}}
\begin{tabular}{@{\,}r <{\hskip 2pt} !{\foo} >{\raggedright\arraybackslash}p{9cm} !{\fooo} >{\raggedright\arraybackslash}p{5cm}} 
\multicolumn{1}{@{\,}r <{\hskip 9pt}}{DATE} & \multicolumn{1}{l}{THESIS} & \multicolumn{1}{l}{STA4505} \\
\hline
\st{Dec 2014} & \st{Complete CTMC calibration} \\
\st{Dec 2014} & \st{Backtest naive strategies based on CTMC} & \\
\st{Jan-May} & \st{Study stochastic controls: ECE1639, STA4505} & \\
\st{Jun 5} & \st{Establish models} & \st{Exam Study} \\
\st{Jun 12} & \st{Establish performance criteria} & \st{Exam Study} \\
\st{Jun 15} & \st{Derive DPP/DPE} & \st{EXAM} \\
\st{Jun 19} & \st{Derive DPP/DPE} &  \\
\st{Jun 26} & \st{Derive continuous-time equations} & \\
\st{Jul 3} & \st{Derive discrete-time equations} & \\
\st{Jul 10} & \st{Set up MATLAB numerical integration} &  \\
\st{Jul 17} & \st{Integrate functions and plot dynamics} & \st{Integrate and analyze} \\
Jul 24 & More dynamics, and calib/choose parameters & \\
Jul 31 & Backtest on historical data & \st{Simulate results} \\
Aug 7 & Backtest, comparing with naive strategies & \\
Aug 14 & Dissertation writeup / buffer & Project writeup \\
Aug 21 & Dissertation writeup / buffer &  \\
Aug 28 & Dissertation writeup & Presentation \\
\end{tabular}
\end{table}

\newpage
\section*{Whiteboard Inspirational Quote of the Week}
\begin{quote}
\textit{Solving a problem for which you know there's an answer is like climbing a mountain with a guide, along a trail someone else has laid. In mathematics, the truth is somewhere out there in a place no one knows, beyond all the beaten paths. And it's not always at the top of the mountain. It might be in a crack on the smoothest cliff or somewhere deep in the valley.} \\-- Yoko Ogawa, \textit{The Housekeeper and the Professor}
\end{quote} 

\section*{For Our Readers in the Middle East...}
my friend...it's been quite the week, a weak week but still a week, and I'm sorry that contained here within is perhaps the most disappointing TMW to date. a mixed bag of responsibility for this state of affairs - on the one hand i spent a considerable amount of time doing pan am games stuff, which i have absolutely no regrets about; and on the other i had that really disappointing realization that shit was fucked which took about two full days out, largely just trying to destress and await further instruction from the overlord sebastian himself. so as mentioned we're back on trajectory by just fudging things a little bit, and one way or another this'll produce some solid results.

\section*{The Academic Week in Review}
This week we develop numerical solutions for the simplified dynamic programming equations, and thus solve for the unknown ansantz functions $h$. Generally speaking, we will use an explicit finite difference scheme to solve the differential equations, and a cutoff method for evaluating the the quasi-variational inequalities. 

TODO: Elaborate on these methods in greater detail. Is that redundant? 

So look all I've got so far is code producing plots for my delta functions that I have virtually no way of assessing for correctness. There's definitely `funny' behaviour that I can't quite explain. With respect to where we want to be on the progress timeline, I've numerically solved and plotted cases 1 \& 2 in continuous time, which leaves cases 3 as well as the discrete time case. 

Seems like it may not be correct to delineate the numerical solutions and simulation/backtesting, as it's proving difficult to make any sense of the plots. It would lend some credibility to the results if I could run them through a backtest and see what sort of results they produce... if they're multi-million dollar losses then I know I gotta flip a minus sign somewhere...

So, the plan for the way forward: finish the code for case 3 and discrete time (by EOW), develop backtesting capability (EOTu), spend the rest of the week going back and forth with the results. 
\end{document}
