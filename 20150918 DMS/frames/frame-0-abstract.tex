\begin{frame}[plain]
\frametitle{Abstract}
\begin{quote}
{\scriptsize
This dissertation demonstrates that there is high revenue potential in using limit order book imbalance as a state variable in an algorithmic trading strategy. Beginning with the hypothesis that imbalance of bid/ask order volumes is an indicator for future price changes, exploratory data analysis suggests that modelling the joint distribution of imbalance and observed price changes as a continuous-time Markov chain presents a monetizable opportunity. The arbitrage problem is then formalized mathematically as a stochastic optimal control problem using limit orders and market orders with the aim of maximizing terminal wealth. The problem is solved in both continuous and discrete time using the dynamic programming principle, which produces both conditions for market order execution, as well as limit order posting depths, as functions of time, inventory, and imbalance. The optimal controls are calibrated and backtested on historical NASDAQ ITCH data, which produces consistent and substantial revenue.}
\end{quote}
\end{frame}

%{
%    \usebackgroundtemplate{\includegraphics[height=\paperheight,width=\paperwidth]{frames/figs/wordforword.jpg}}
%    \setbeamertemplate{navigation symbols}{}
%    \begin{frame}[plain]
%    \end{frame}
%    }