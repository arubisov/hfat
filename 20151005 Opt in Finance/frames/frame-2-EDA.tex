% introduce CTMC 
\begin{frame}
\frametitle{Limit-Order Book Imbalance}
\emph{Imbalance} is a ratio of quoted limit order volumes between the bid and ask side.
\[ I(t) = \frac{V_{bid}(t) - V_{ask}(t)}{V_{bid}(t) + V_{ask}(t)} \in [-1,1] \]
\end{frame}

\begin{frame}
\frametitle{Limit-Order Book Imbalance}
\only<1>{\adjustimage{max size=\textwidth}{frames/figs/LOB4.pdf}}
\only<2>{\adjustimage{max size=\textwidth}{frames/figs/LOB4-formula.pdf}}
\end{frame}

\begin{frame}
\frametitle{Limit-Order Book Imbalance}
\only<1>{
Resulting timeseries is noisy.
\adjustimage{max size=\textwidth}{frames/figs/imbalance1.pdf}
}
\only<2>{
Smooth it by averaging on a sliding window (1 s).
\adjustimage{max size=\textwidth}{frames/figs/imbalance2.pdf}
}
\only<3>{
Model as a continuous-time Markov chain $Z(t)$ with generator $G$.
\adjustimage{max size=\textwidth}{frames/figs/imbalance3.pdf}
}
\only<4>{
Model as a continuous-time Markov chain $Z(t)$ with generator $G$.
\vspace{0.1\textwidth}%
\[ \def\arraystretch{2}%
Z = \left\lbrace \begin{array}{lll}
5, & \rho \in [+\frac{3}{5},+1], & \text{buy-heavy} \\
4, & \rho \in [+\frac{1}{5},+\frac{3}{5}], & \text{buy-biased} \\
3, & \rho \in [-\frac{1}{5},+\frac{1}{5}), & \text{neutral} \\
2, & \rho \in [-\frac{3}{5}, -\frac{1}{5}), & \text{sell-biased} \\
1, & \rho \in [-1, -\frac{3}{5}), & \text{sell-heavy} 
\end{array} \right. \]
\vspace{0.1\textwidth}%
}
\end{frame}

% introduce nFPC calibration   
\begin{frame}
\frametitle{Incorporating Price Change}
Next, consider a two-dimensional CTMC $Z(t)$ that jointly models imbalance bin $\rho(t)$ and price change $\Delta S(t)$, where 
\[ \rho(t) \in \lbrace 1,2,\dots,\#_{bins} \rbrace \]
is the bin corresponding to imbalance averaged over the interval $[t-\Delta t_I, t]$, and
\[ \Delta S(t) = \sgn(S(t+\Delta t_S)-S(t)) \in \lbrace -1, 0, 1 \rbrace \]  
is the \emph{sign} of the change in midprice of the \emph{future} time interval $\Delta t_S$.

\adjustimage{max size=\textwidth}{frames/figs/timewindows.pdf}
\end{frame}

% P matrix.
\begin{frame}
\frametitle{Transition Matrix}
Using MLE, we obtain a generator matrix $\mat{G}$ for the CTMC. The transition matrix over a step of size $\Delta t_I$ is given by
\[ \mat{P}(\Delta t_I) = [p_{ij}(\Delta t_I)] = e^{\mat{G} \Delta t_I} \]
called our \emph{one-step transition probability matrix}. Matrix entries give the probability of transition from one (imbalance, price change) pair to another over the time interval $\Delta t_I$. This can be written semantically as
\[ p_{ij} = \mathbb{P}\left[ \varphi( \rho_\text{curr}, \Delta S_\text{future}) = j \; | \; \varphi( \rho_\text{prev}, \Delta S_\text{curr} ) = i \right] \]
\end{frame}

% ... Bayes transform into Q matrix ...
\begin{frame}
\frametitle{Predicting Future Price Change}
Using Bayes' Rule, we can transform the $\mat{P}$ matrix to 
\[
\resizebox{\textwidth}{!}{$\displaystyle
\mathbb{P}\left[ \Delta S_\text{future} = j \; | \; \substack{
\rho_\text{curr} = i \\
\rho_\text{prev} = k\\
\Delta S_\text{curr} = m} \right] = \dfrac{\mathbb{P}\left[ \rho_\text{curr} = i, \Delta S_\text{future} = j \; | \; \substack{
\rho_\text{prev} = k\\
\Delta S_\text{curr} = m} \right]}{\mathbb{P}\left[ \rho_\text{curr} = i \; | \; \substack{
\rho_\text{prev} = k\\
\Delta S_\text{curr} = m} \right]}
$}
\]
This allows us to predict future price moves.\par
\vspace{\baselineskip}
We'll call the collection of these probabilities the $\mat{Q}$ matrix.
\end{frame}

% show sample Q. 
\begin{frame}
\frametitle{Predicting Future Price Change}
Sample $\mat{Q}$ matrix calibrated on \texttt{MMM}, 2013-05-15.
\begin{table}%
\ra{1.2}%
\begin{adjustbox}{width=\textwidth}
\begin{tabular}{@{} r@{\hskip 1cm} *{9}{r} @{}}%
\toprule
& \multicolumn{3}{c}{$\Delta S_\text{curr} < 0$} & \multicolumn{3}{c}{$\Delta S_\text{curr} = 0$} & \multicolumn{3}{c}{$\Delta S_\text{curr} > 0$} \\
\cmidrule(lr){2-4} \cmidrule(lr){5-7} \cmidrule(lr){8-10}
\multicolumn{2}{r}{$\rho_{curr} = 1$} & 2 & 3 & 1 & 2 & 3 & 1 & 2 & 3 \\
\midrule
\multicolumn{10}{l}{$\Delta S_\text{future} < 0$} \\
$\hphantom{abcd}\rho_\text{prev} = 1$ & \bf 0.53 & 0.15 & 0.12 & 0.05 & 0.10 & 0.14 & 0.08 & 0.13 & 0.14 \\
$\rho_\text{prev} = 2$ & 0.10 & \bf 0.58 & 0.14 & 0.07 & 0.04 & 0.10 & 0.13 & 0.06 & 0.12 \\
$\rho_\text{prev} = 3$ & 0.08 & 0.12 & \bf 0.52 & 0.09 & 0.06 & 0.03 & 0.11 & 0.10 & 0.05 \\[0.6ex]
\multicolumn{10}{l}{$\Delta S_\text{future} = 0$} \\
$\rho_\text{prev} = 1$ & 0.41 & 0.75 & 0.78 & \bf 0.91 & 0.84 & 0.79 & 0.42 & 0.79 & 0.77 \\
$\rho_\text{prev} = 2$ & 0.79 & 0.36 & 0.71 & 0.83 & \bf 0.92 & 0.82 & 0.75 & 0.37 & 0.78 \\
$\rho_\text{prev} = 3$ & 0.79 & 0.74 & 0.40 & 0.81 & 0.83 & \bf 0.91 & 0.70 & 0.76 & 0.39 \\[0.6ex]
\multicolumn{10}{l}{$\Delta S_\text{future} > 0$} \\
$\rho_\text{prev} = 1$ & 0.06 & 0.10 & 0.09 & 0.04 & 0.06 & 0.07 & \bf 0.50 & 0.09 & 0.09 \\
$\rho_\text{prev} = 2$ & 0.10 & 0.06 & 0.15 & 0.10 & 0.04 & 0.08 & 0.12 & \bf 0.57 & 0.10 \\
$\rho_\text{prev} = 3$ & 0.13 & 0.14 & 0.08 & 0.10 & 0.11 & 0.05 & 0.19 & 0.14 & \bf 0.56 \\
\bottomrule
\end{tabular}%
\end{adjustbox}%
\end{table}%
\end{frame}
%
%% present Naive strategies 1-3
%\begin{frame}
%\frametitle{Trading Strategies Informed by the $\mat{Q}$ Matrix}
%\setlength{\leftmargini}{50pt}
%\begin{itemize}
%\item[Naive] Use market orders to buy (sell) if price is predicted to move up (down).
%\item[Naive+] Post at-the-touch limit orders when zero price change is predicted.
%\item[Naive++] Post a limit order to buy (sell) is price is predicted to move up (down).
%\end{itemize}
%\end{frame}
%
%\begin{frame}
%\frametitle{Calibration of Naive Trading Strategies}
%Need to select:
%\begin{itemize}
%\item price change observation period $\Delta t_S$
%\item imbalance averaging period $\Delta t_I$
%\item number of imbalance bins $\#_{bins}$
%\end{itemize}
%Calibration done on the first day of the trading year, same parameters used for all days.
%
%Brute-force search of parameter space, using max Sharpe ratio criterion, found that $\Delta t_S = \Delta t_I = 1 sec$, and $\#_{bins} = 4$
%\end{frame}
%
%% present plots
%\begin{frame}
%\frametitle{Results of Naive Trading Strategies}
%\begin{figure}%
%\centering%
%\subfloat{\includegraphics[height=.3\textwidth, width=.35\textwidth]{frames/figs/FARO_naive_strat_comp.pdf}}\quad
%\subfloat{\includegraphics[height=.3\textwidth, width=.35\textwidth]{frames/figs/NTAP_naive_strat_comp.pdf}}\\
%\subfloat{\includegraphics[height=.3\textwidth, width=.35\textwidth]{frames/figs/ORCL_naive_strat_comp.pdf}}\quad
%\subfloat{\includegraphics[height=.3\textwidth, width=.35\textwidth]{frames/figs/INTC_naive_strat_comp.pdf}}\\
%%
%\leavevmode\smash{\makebox[0pt]{\hspace{2em}% HORIZONTAL POSITION           
%  \rotatebox[origin=l]{90}{\hspace{3em}% VERTICAL POSITION
%	Normalized Profit and Loss (P\&L)}%
%}}\hspace{0pt plus 1filll}\null%
%
%Time (h)
%
%\vspace{1cm}%
%\subfloat{\includegraphics[scale=1]{frames/figs/naivestratlegend.pdf}}%
%\end{figure}
%
%\end{frame}
%
%% talk about calibration and failures
%\begin{frame}
%\frametitle{Conclusions from Naive Trading Strategies}
%{\bf Why is the Naive strategy producing, on average, normalized losses?}
%\begin{itemize}
%\item Backtest is out-of-sample; evidence to reject time-homogeneity
%\item Calibration is done on first trading day; likely nonrepresentative of trading activity
%\item Price change probability matrix $\mat{Q}$ obtained using midprices, ignoring bid-ask spread; $\sgn(\Delta S)$ may be insufficient for create profit, especially on \texttt{FARO}
%\end{itemize}
%\end{frame}
%
%\begin{frame}
%\frametitle{Conclusions from Naive Trading Strategies}
%{\bf Why do the Naive+ and Naive++ strategies outperform the Naive strategy?}
%\begin{itemize}
%\item LOs vs MOs means different transaction price is being used (only MO loses value)
%\item Naive only executes when predicting non-zero price change
%\begin{itemize}
%\item Only sign, not magnitude
%\item Only \emph{if one was already seen}
%\end{itemize}
%\end{itemize}
%\end{frame}
